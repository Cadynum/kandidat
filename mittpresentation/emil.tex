\section{Översikt och Mål}
\subsection{Systemöversikt}
\begin{frame}
\includegraphics[width=\textwidth]{../mittrapport/img/systemchart}
\end{frame}

\subsection{Mål}
\begin{frame}
Handen ska kunna
\begin{itemize} 
\item gripa och lyfta ett mjölkpaket med en vikt av 1 kg.
\item gripa och lyfta en mutter av storlek M10 mellan tumme och pekfinger.
\item klara att att lyfta en last motsvarande ett kg på mitten av ett finger.
\item lyfta upp en penna.
\item lyfta upp en snusdosa.
\end{itemize}
\includegraphics[width=\textwidth]{../mittrapport/img/provagrepp}
\end{frame}

\begin{frame}
\begin{itemize}
\item Fingerspetsen ska kunna inta två olika lägen utan att flytta handen. (trycka på två olika knappar)
\item Information om trycket som handen påverkar objektet med ska mätas.
\item Handen skall kunna kontrollera trycket som den applicerar på objekt.
\item En ovan användare skall efter kalibrering kunna utföra ovanstående mål.
\item Maximal tid för handen att röra sig från maximalt öppen till en knuten näve är en sekund.
\end{itemize}
\end{frame}


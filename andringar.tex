\section{Uppföljning av planeringsrapporten}
När projektet befann sig i ett tidigare stadie hölls det abstrakt för att vara öppet för så många olika lösningar som möjligt. När arbetet inleddes ändrades karaktären till att vara mer målbestämt med tydligare detaljer om hur arbetet skulle utföras.

De mekatroniska delmålen avklarades snabbare än mjukvarudelmålen eftersom det inte behövdes beställa lika många nya komponenter för att tillverka mekaniken. Den mekaniska framtagningen främjades även av möjligheten att experimentera med existerande delar utan att behöva motivera inköp med beräkningar. 

Enligt tidsrapporten skulle stort fokus läggas på framtagning av de olika koncepten. När processen väl började framgick det snabbt att ett koncept visade sig vara överlägset alla andra. Det gjorde att ett konceptval snabbt kunde göras och att vidareutvecklingen startade tigigare. Utvecklingsarbetet av detaljerna visade sig ta längre tid eftersom många delars utformning påverkade den slutgiltiga produktens funktion. För att den skulle klara att utföra de uppsatta målen krävdes att delarna utformades noga 
\section{Sammanfattning}

Följande är en rapport som beskriver  "Design av robothand" som går ut på att konstruera en mekanisk hand som styrs trådlöst med en styrhandske, där användarens rörelser efterliknas av robothanden. Robothanden känner även av hur mycket de greppade föremålet påverkas med hjälp av trycksensorer vilket ger en återkoppling av handens gripkraft.

Med utgångspunkt i liknande arbeten framtas en grundläggande mekanisk princip  som används för att konstruera de separata fingrarna. För att manipulera fingrarna används sex stycken hobbyservon som kontrollerar fingrarna via senor och stag. För att få en uppfattning (om möjligt, hitta bättre ord än “uppfattning”) om handens utseende, funktionalitet och rörelse används CAD-ritningar och rörelsesimuleringar innan tillverkning. 

Kommunikationen mellan styrhandsken och robothanden upprättas med hjälp av mikrokontrollers, så kallade Arduinokort. Signalerna från styrhandsken reglerar robothandens rörelse och fås från flexgivare* som sitter på styrhandsken.
Mikrokontrollerna styr även de hobbyservon som i sin tur får robothanden att röra sig.
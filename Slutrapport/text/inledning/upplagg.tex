
\section{Rapportens Upplägg}
'' "Report outline" describes very short what is presented in each of the following chapters. From this outline it should be clear which chapters give a theoretical backgrond, which chapter describe tools you use, which describe your contribution, and which describes your results.''-Jonas

Rapporten är upplagd som följer:
I kapitel 2 ges en beskrivning av systemets komponenter och teorin bakom valet av dem. Detta innefattar den mekaniska designen av robothanden, styrhandsken och färdiga komponenter som används. I kapitel 3 redogörs vilka algoritmer och filter som utvecklats för att uppnå styrning och reglering samt resultat av dessa implementerade med hela konstruktionen. Kapitel 4 redogör för diskussion kring resultaten i kapitel 3. 

%Projektdelen borde vara i en avskild del "del 2" t.ex med egna kapitel, efter appendix?

%Kapitel 1 2 3 är upplagd som en teknisk rapport där fokus ligger på utförandet och de verktyg som använts.

%Kapitel 4 och framåt är projektinriktat och fokuserar på hur projektet utfördes


%\section{Avgränsningar} Detta ska smältas ihop med inledning
%Budgeten för prototypen är 5000 kr vilket begränsar komplexiteten och antal frihetsgrader. Robothanden har därför endast 2 fingrar och en tumme med sammanlagt åtta frihetsgrader varav två är tvångsstyrda. Skelettet till robothanden har byggts i meccano (ref till förklaring av meccano) vilket begränsar designen till standardiserade mått och begränsar stabiliteten i kontruktionen. Robothanden är försedd med tre trycksensorer vilket avsevärt begränsar antal grepp som möjliggör objektidentifiering.

 
 



%\section{Vad inledningen ska innehålla enligt anvisningar}
%Inledningen sätter in rapporten i ett sammanhang och visar dess
%relevans och nyhetsvärde. Den fungerar som en introduktion till
%hela rapporten och ska ge läsaren nödvändig information som
%behövs för att ta del av dess innehåll.
%Inledningen innehåller normalt en syftesformulering som ofta
%ställs i relation till en bakgrund eller kort historik. I många fall
%är syftesformuleringen nära relaterad till den
%problemformulering som är viktig för att såväl läsare som
%skribent ska kunna utnyttja rapporten väl. Det bör också stå
%något om undersökningens eller experimentets omfattning och
%anledningar till särskilda avgränsningar. Vidare bör också
%metod finnas med, men endast i syfte att ange vilken typ av
%undersökning som gjorts. Metoden utvecklas i andra avsnitt av
%rapporten.
%Man brukar ange bakgrund, syfte och metod som inledningens
%obligatoriska funktioner. Ibland signaleras även centrala resultat
%redan i inledningen.
%Inledningen är den första sidan som sidnumreras.



%1 http://www.shadowrobot.com/products/dexterous-hand/

%2 http://www.nytimes.com/2013/03/30/science/making-robots-mimic-the-human-hand.html?_r=0

%ExoHand 3 http://www.forbes.com/sites/singularity/2012/07/06/sophisticated-robotic-hand-also-doubles-as-a-human-exoskeleton/

%(Cutkosky och Howe, 1990)
%http://biorobotics.harvard.edu/pubs/1990/human%20grasp%20choice.pdf

%Grepp identifiering. Kan kanske bli användbart vid objektidentifieringssnack http://cg.cis.upenn.edu/hms/research/RIVET/graspTypeRecog.pdf
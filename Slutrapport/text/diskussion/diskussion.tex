\chapter{Diskussion och rekommendationer}
Testresultaten av objektidentifieringen visar på brister i systemet.


\section{Mekanisk konstruktion}
Den mekaniska konstruktionen fungerar väl vad gäller rörelseomfång och flexibilitet. I figur~\ref{fig:greppagrap} demonstreras att robothanden klarar av att utföra greppen som var vägledande för designen, se~\ref{cutshand}. Dock har Meccano höga toleranser vilket leder till glapp i axlar och lederna. Detta glapp uppträder olika beroende på hur de sex servomotorerna är positionerade, vilket gör att det är svårt att kompensera för glappet i den matematiska modellen för att förbättra objektidentifieringen. Det rekommenderas att leder och axlar rekonstrueras med finare toleranser, alternativt att Meccano slopas och detaljer egentillverkas.\\ Fingertopparna designades för att kunna utföra pincettgrepp där kontaktrycket mäts. Pincettgreppet fungerar väl men det visar sig att fingertoppens platta yta gör det svårt för sensorn att registrera tryck vid hantering av platta objekt. En mer rundad design kommer göra att trycksensorn sticker ut mer ur fingertoppen och blir mer utsatt vid hantering av olika objekt, och därmed enklare kommer registrera kontakttryck. Då robothanden inte bara utnyttjar fingertopparna för att hantera objekt, bör möjligheten att uppskatta krafter som uppstår mellan objekt och övriga delar av handen utredas.



Ett problemen med styrning av robothanden och dess precision vid identifiering av objekt är stabiliteten i konstruktionen. Att bygga i meccano resulterade i glapp mellan olika delar och töjningar av materialet som inte räknades med i den matematiska modellen. Detta gör att en korrekt bedömning av avståndet mellan fingertopparna för att identifiera objekt får låg precision varför antalet objekt som kan implementeras i programmet kraftigt begränsas. Problem med skruvar som lossnar och delar som blir sneda gjorde dessutom att styrningsresultat kunde skilja sig från en gång till en annan. Detta innebär att en kalibrering av servomotorernas läge för greppning av ett bestämt objekt behövdes oftare än tänkt.
För vidareutveckling rekommenderas en stabilare konstruktion där återkoppling av servovinklar utvecklas där man sedan kan återanvända den nuvarande mjukvaran.

Ett annat problem är avläsning av användarens rörelse i styrhandsken. Den nuvarande styrhandsken använder flexsensorer som reagerar på en böjning av själva handsken. Problemet är att handsken följer inte handen rörelse exakt eftersom det blir glapp mellan hand och handske. Detta medför att avläsningen inte blir kontinuerlig och exakt. För att lösa detta rekommenderas att utveckla en handske som följer handens exakta rörelse likt Festo's Exohand som nämns i inledningen. De använder en styrhandske som ser likadan ut som robothanden vilket ger mer exakt och intuitiv styrning.

\section{Möjliga förbättringar av systemets latenstid}
Den största möjligheten till förbättring är, bortsett från snabbare servomotorer, en annan typ av trådlös kommunikation. 30 ms är lång tid för att en trådlös signal att propagera 10 m i ett öppet rum. Ett möjligt alternativ kan vara att specificera ett eget protokoll med en lågnivå\footnote{I bemärkelsen att man kontrollerar hårdvaran direkt.} radio-transceiver\footnote{Transmitter/receiver --- sändare och mottagare}. Pingtider runt 1 ms borde då vara realistiskt att uppnå.

Bättre kvalité på flexsensorer och trycksensorer kan tänkas leda till mindre störningar, vilket gör att filtrets brytfrekvens kan ökas och därmed minska dess latenstid. Mikrokontrollens frekvens kan även ökas, men till en kostnad av högre strömförbrukning.


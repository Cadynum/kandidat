\chapter{Diskussion och rekommendationer}
Testresultaten av objektidentifieringen visar på brister i systemet. Då det är många områden i systemet som inverkar på dess totala funktion, finns ett flertal tänkbara felkällor. Nedan diskuteras detta och rekommenderade förbättringar presenteras.


\section{Mekanisk konstruktion}
Den mekaniska konstruktionen fungerar väl vad gäller rörelseomfång och flexibilitet. I figur~\ref{fig:greppagrap} demonstreras att robothanden klarar av att utföra greppen som var vägledande för designen, se appendix~\ref{cutshand}. Dock har Meccano höga toleranser vilket leder till glapp i axlar och lederna. Detta glapp uppträder olika beroende på hur de sex servomotorerna är positionerade, vilket gör att det är svårt att kompensera för glappet i den matematiska modellen för att förbättra objektidentifieringen. Det rekommenderas att leder och axlar rekonstrueras med finare toleranser, alternativt att Meccano slopas och detaljer egentillverkas.\\ Fingertopparna designades för att kunna utföra pincettgrepp där kontaktrycket mäts. Pincettgreppet fungerar väl men det visar sig att fingertoppens platta yta gör det svårt för sensorn att registrera tryck vid hantering av platta objekt. En mer rundad design kommer göra att trycksensorn sticker ut mer ur fingertoppen och blir mer utsatt vid hantering av olika objekt, och därmed enklare kommer registrera kontakttryck. Då robothanden inte bara utnyttjar fingertopparna för att hantera objekt, bör möjligheten att uppskatta krafter som uppstår mellan objekt och övriga delar av handen utredas.

\section{Objektidentifiering}
Den implementerade objektidentifieringen kan förbättras. Den kan utökas till att vara tredimensionell för att bättre kunna särskilja olika objekt. den matematiska modellen som beskriver handen överensstämmer med den konstruerade handens ideala mått, men tar som sagt inte hänsyn till det mekaniska glappet. Avståndsberäkningarna utgår i dagsläget från de önskade servovinklarna. Detta medför att glappet samt tidsfördröjningen som uppstår mellan att användaren begär en förändring av servoläget tills dess att servomotorn uppåntt det nya läget, inte tas med i beräkningen av avståndet. En förbättring av detta vore att mäta det faktiska servoläget och/eller fingrarnas faktiska vinklar. Om detta görs kan återkoppling av de faktiska vinklarna i systemet göras, vilket kan utnyttjas i ett reglersystem. Detta reglersystem kan se till att fingrarna når de önskade positionerna, samt uttnyttjas för att justera fingrarna då den uppmätta tryckkraften vid hantering av objekt går utanför det önskvärda. Vid testet av begränsing av tryck stoppas vidare aktuering av robothanden då det önskade trycket överskrids. Dock överskrider trycket ändå det maximala, och vid ett senare skede där det greppade objektet hoppar till i handens grepp sjunker den uppmätta tryckkraften momentant under gränsvärdet vilket får handen att röra sig mot användarens begärda värde och trycket överstiger nu kraftigt det högsta tillåtna. Detta test visar att det inte är tillräckligt att endast stoppa vidare aktuering för att förhindra att objekt påverkas av för stora krafter, utan detta måste styras med ett aktivt reglersystem, som kan justera greppet så att krafterna inte överstiger det högsta tillåtna.

\section{Styrhandske}

Ett annat problem är avläsning av användarens rörelse i styrhandsken. Den nuvarande styrhandsken använder flexresistorer som reagerar på en böjning av själva handsken. Problemet är att handsken inte följer handens rörelse exakt då det finns glapp mellan hand och handske. Detta medför att avläsningen inte blir exakt.  Ytterligare en faktor som påverkar styrningen är hur flexgivarna är placerade på handsken. För att robothanden ska röra sig likt användarens hand är det viktigt att flexgivarna sitter placerade så att de registrerar samtliga rörelser entydigt. Då alla flexresistorer sitter på samma handske påverkar rörelser i ett finger de andra då handsken dras med i fingerrörelsen vilket gör att övriga fingrars flexresistorer ändrar värde. En rekommendation är att mäta användarens fingerrörelser på ett sådant sätt att alla rörelser frikopplas och inte kan påverka varandra.

\section{Möjliga förbättringar av systemets latenstid}
Den största möjligheten till förbättring är, bortsett från snabbare servomotorer, en annan typ av trådlös kommunikation. 30 ms är lång tid för att en trådlös signal att propagera 10 m i ett öppet rum. Ett möjligt alternativ kan vara att specificera ett eget protokoll med en lågnivå\footnote{I bemärkelsen att man kontrollerar hårdvaran direkt.} radio-transceiver\footnote{Transmitter/receiver --- sändare och mottagare}. Pingtider runt 1 ms borde då vara realistiskt att uppnå.

Bättre kvalité på flexsensorer och trycksensorer kan tänkas leda till mindre störningar, vilket gör att filtrets brytfrekvens kan ökas och därmed minska dess latenstid. Mikrokontrollens frekvens kan även ökas, men till en kostnad av högre strömförbrukning.


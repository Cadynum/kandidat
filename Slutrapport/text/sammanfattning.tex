

Följande är en rapport som beskriver utförandet av ''Design av robothand'', ett arbete som handlar om att konstruera en mekanisk hand som styrs trådlöst med en styrhandske, där användarens rörelser efterliknas av robothanden. Robothanden känner även av hur mycket de greppade föremålet påverkas med hjälp av trycksensorer vilket ger en återkoppling av handens gripkraft.

Med utgångspunkt i liknande arbeten framtas en grundläggande mekanisk princip  som används för att konstruera de separata fingrarna. För att simulera handens utseende och funktionalitet innan tillverkning används CAD-ritningar och matematiska modeller av fingrarna.

För att manipulera fingrarna används sex stycken hobbyservon på vardera 6 volt som styr fingrarna via senor och stag.

Kommunikationen mellan styrhandsken och robothanden upprättas med hjälp av bluetooth och Arduinomikrokontrollers. Signalerna från styrhandsken reglerar robothandens rörelse och fås från flexgivare som sitter på styrhandsken.
Mikrokontrollerna styr även de hobbyservon som i sin tur får robothanden att röra sig. 

Mer om trycksensorerna när vi vet mer..
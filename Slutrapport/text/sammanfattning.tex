\section*{Sammanfattning}
Denna rapport beskriver en robothand med tre fingrar som styrs trådlöst med en handske. Robothanden är byggd av Meccano och har totalt åtta rörliga leder varav sex är separat styrbara och aktueras av servomotorer. Handen har även trycksensorer på fingertopparna för att ge användaren feedback med diodramper om hur hårt objektet påverkas. Kommunikationen mellan styrhandsken och robothanden upprättas med hjälp av bluetooth och Arduinomikrokontrollers. Signalerna från styrhandsken reglerar robothandens rörelse och fås från flexgivare som sitter på styrhandsken. För att beräkna avståndet mellan fingertopparna och därmed möjliggöra objektidentifiering används en matematisk modell av handen. Genom att beräkna avståndet kan ett antal fördefinerande objekt identifieras. Objektidentifieringen används sedan för att begränsa kontakttrycket som objektet påverkas av.

Funktionstester där olika objekt greppas demonstreras tillsammans med objektidentifieringen och tryckbegränsning. Resultaten utvärderas och möjliga förbättringar ges i ett avslutande diskussionsavsnitt.

\section*{Abstract}
This report describes a robothand with three fingers that are controlled wirelessly with a glove.
The robothand is built with Meccano and has a total of eight movable joints of which six are
separately controllable and actuated by servomotors. The hand also has pressure sensors on the fingertips
to give the user feedback with diodbars of how hard the object is affected.
The communication between the glove and the robothand is established by using bluetooth
and Arduino microcontrollers. Signals from the glove controls the robothands motion and are obtained from the flex sensors on the glove.
To compute the distance between the fingertips and thus enable object identification, a mathematical model is used of the robothand. By calculating the distance a number of defined items can be identified. Object identification is then used to
limit contact pressure that the object is affected by.
Functional tests in which different objects were grasped is demonstrated with object identification and pressure limit. The results are evaluated and possible improvements are given in a concluding discussion section.











\thispagestyle{plain}
\vspace*{\fill}
\section*{Sammanfattning}
Denna rapport redogör utveckling och konstruktion av en robothand med tre fingrar, vilken styrs trådlöst via en handske som användaren bär. Handsken kalibreras mot olika användares händer och mäter kontinuerligt fingrarnas läge. Robothanden är försedd med trycksensorer som visuellt återkopplar aktuellt tryck till användaren via dioder på handsken. Med matematisk modellering av robothanden tillsammans med styrsignaler till aktuatorerna uppnås en enkel objektidentifiering baserat på objektets storlek.
Om ett objekt identifierats och användaren överskrider ett fördefinierat tryck begränsas handen från att greppa hårdare, och på så vis skyddas objektet från skada.
För att inte små mätstörningar ska påverka tryckbegränsningen och för att styrningen inte ska upplevas hackig filtreras signalerna. Robothandens design är framtagen dels för att uppnå intuitiv styrning, och även för att utvalda grepp ska kunna utföras.

Användartester visar att robothandens styrning och tryckåterkoppling upplevs intuitiv. Robothanden identifierar och begränsar greppkraften korrekt, men skillnader mellan den idealiserade modellen och den verkliga rörelsen begränsar antal och storlek på objekt som kan identifieras. Felkällor finns främst i den mekaniska konstruktionen på grund av höga toleranser.

% Kommunikationen mellan styrhandsken och robothanden upprättas med hjälp av bluetooth och arduinomikrokontrollers.

%Funktionstester där olika objekt greppas demonstreras tillsammans med objektidentifieringen och tryckbegränsning. Resultaten utvärderas och möjliga förbättringar ges i ett avslutande diskussionsavsnitt.

\section*{Abstract}
This report describes a robothand with three fingers that are controlled wirelessly with a glove.
The robothand is built with Meccano and has a total of eight movable joints of which six are
separately controllable and actuated by servomotors. The hand also has pressure sensors on the fingertips
to give the user feedback with diodbars of how hard the object is affected.
The communication between the glove and the robothand is established by using bluetooth
and Arduino microcontrollers. Signals from the glove controls the robothands motion and are obtained from the flex sensors on the glove.
To compute the distance between the fingertips and thus enable object identification, a mathematical model is used of the robothand. By calculating the distance a number of defined items can be identified. Object identification is then used to
limit contact pressure that the object is affected by.
Functional tests in which different objects were grasped is demonstrated with object identification and pressure limit. The results are evaluated and possible improvements are given in a concluding discussion section.
\vspace*{\fill}


\section{Elektriska kretsar}
\begin{figure}[H]
\includegraphics[height=0.5\textheight]{img/schemahandske}
\caption{Kopplingsschema för styrhandsken.}
\label{schemahandske}
\end{figure}

\begin{figure}[H]
\includegraphics[height=0.5\textheight]{img/schemahand}
\caption{Kopplingsschema för robothanden.}
\label{schemahand}
\end{figure}


Komponenterna för styrhandsken är lödda på ett experimentkort medan robothanden är direktkopplad. Arduino kan maximalt ge 5 Volt och är begränsad till små strömmar.  
För att strömförsörja robothanden används ett 7.4 Volts LiPo-batteri och Arduino Due försörjs av ett 9 Volts batteri, anledningen till att de matas separat är för att undvika ett spänningsfall hos batteriet som kan skada Arduinon. 

Styrhandsken strömförsörjs av ett batteri som är reglerat till 5 Volt.
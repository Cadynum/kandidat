\section{Styrhandsken}
\begin{figure}[H]
\includegraphics[width=1\textwidth]{img/schemahandske}
\caption{Kretsschema för styrhandsken.}
\label{schemahandske}
\end{figure}
Komponenterna till styrhandsken är lödda på ett experimentkort tillsammans med Arduinon och bluetoothmodulen, och strömförsörjs av ett batteri som är reglerat till 5 Volt. $R_1=\unit{3.3}{k\Omega}, R_2=\unit{5}{k\Omega}. $

\subsection{Diodramp}
\label{apx:diodramp}
\begin{figure}[H]
\includegraphics[height=0.5\textheight]{img/diodramp}
\caption{Kretsschema för en av tre diodramper.}
\end{figure}
I figuren ovan är ett kopplingsschema för en diodramp som sitter på styrhandsken. Det finns tre stycken men kopplingsschemat ser likadant ut. \emph{DATA} är kopplad till mikrokontrollen, \emph{MR} är reset och \emph{CLK} är klockpulsen. $R_1=\unit{180}{\Omega}$
\section{Robothanden}
\begin{figure}[H]
\includegraphics[width=1\textwidth]{img/schemahand}
\end{figure}


\comment{öjeling kommenterar: detta borde vara under löptext i rapporten, som refererar till appendix}
Robothandens komponenter är direktkopplade utan experimentkort. Arduinon kan maximalt ge 5 Volt och är begränsad till 800mA så för att strömförsörja robothanden används ett 7.4 Volts LiPo-batteri och Arduino Due försörjs av ett 9 Volts batteri. De matas separat är för att undvika ett spänningsfall hos batteriet som kan skada Arduinon.

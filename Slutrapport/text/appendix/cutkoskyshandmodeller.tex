\section{Cutkoskys grepphierarki}
\label{cutshand}
Cutkoskys grepphierarki, se figur~\ref{cutkosky}, beskriver systematiskt de vanligaste handgreppen som används vid manuellt handarbete~\citep{Cutkosky}. Den är allmänt accepterad som måttstock då Robothänders flexibilitet och praktiska funktionalitet skall verifieras.
Bland de bästa i dagsläget är Nasa's Robonaut 2 Hand som framgångsrikt kan utföra 90\% av greppen~\citep{Nasa}.

\begin{figure}[H]
\includegraphics[width=\textwidth]{../mittrapport/img/cutkoskys_handmodeller.png}
\caption{Cutkoskys grepphierarki ur~\citep{Cutkosky}}
\label{cutkosky}
\end{figure}

Grepphierarkin användes som hjälpmedel vid designen för att säkerställa att robothanden uppnår flexibilitet och förmåga att hantera olika objekt med olika krav på styrka och finmotorik.

\begin{figure}[H]
\includegraphics[width=0.90\textwidth]{img/provagreppcut}
\caption{Verifiering av utvalda dimensionerande grepp i CAD- miljö.}
\label{fig:testcut}
\end{figure}
I figur~\ref{fig:testcut} ses de fyra greppen som valdes som vägledande för designen och hur de verifieras i CAD innan montrering av den fysiska handen. Greppen går i sin tur att identifiera i figur~\ref{cutkosky} som: fr.v. Heavy wrap 2, 1, Prismatic 9 och Circular 12. Genom att välja grepp som ställer krav på styrka och omfång samt precision och fingerfärdighet säkerställs att robothanden kan utföra ett stort antal grepp.
\chapter{Resultat}
Ett enkelt sätt att verifiera robothandens funktion är att kontrollera objektidentifieringen, samt om robothanden kan stoppa att högre tryck uppstår mellan robothandens fingrar och objektet då det identifieras. Detta test ställer krav på att mekanisk konstruktion, trådlös överföring, sensorer i styrhandske och robothand samt att den matematiska modellen fungerar och kan samverka väl för att uppnå ett gott resultat.

Efter utförda funktionstester har följande uppgifter verifierats. Robothanden kan:
\begin{itemize}
\item Gripa och lyfta ett mjölkpaket med en vikt av ett kg. 
\item Gripa och lyfta en mutter av storlek M10 mellan tumme och pekfinger. 
\item Lyfta en last motsvarande ett kg på mitten av två fingrar.
\item Lyfta upp en snusdosa.
\item Kan sluta sig till knuten näve från öppen hand på under en sekund.
\end{itemize}

\subsection{Objektidentifiering}
För att kunna identifiera objekt väl behöver den matematiska modellen som beskriver hur robothandens fingrar är ställda överensstämma väl med den faktiska konstruktionen av robothanden. 
\begin{figure}[H]
\includegraphics{img/obj_id_matlab2}
\caption{Avståndet mellan det faktiska och beräknade värdet.}
\label{avstand}
\end{figure}

Figur~\ref{avstand} visar hur den matematiska modellen avviker från ett perfekt mätresultat för 18 olika mätningar. Det genomsnittliga felet är 20 procent och inom handens typiska arbetsområde, som är \unit{40-200}{mm}, är felet endast i genomsnitt XX procent med ett största fel på \unit{15}{mm}. Utgående från detta som största felmarginal kan robothanden med säkerhet särskilja objekt som har \unit{15}{mm} differens på det klassificerande måttet.

Med denna väljas två olika testobjekt ut för att kontrollera om robothanden kan särskilja dem. det kan den och alla blir glada och dricker sprit och knullar. THE END
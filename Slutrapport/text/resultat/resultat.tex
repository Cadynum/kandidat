\chapter{Resultat}
Ett enkelt sätt att verifiera robothandens funktion är att kontrollera objektidentifieringen, samt om robothanden kan stoppa att högre tryck uppstår mellan robothandens fingrar och objektet då det identifieras. Detta test ställer krav på att mekanisk konstruktion, trådlös överföring, sensorer i styrhandske och robothand samt att den matematiska modellen fungerar och kan samverka väl för att uppnå ett gott resultat.

\subsection{Objektidentifiering}
För att kunna identifiera objekt väl behöver den matematiska modellen som beskriver hur robothandens fingrar är ställda överensstämma väl med den faktiska konstruktionen av robothanden. 
\begin{figure}[H]
\includegraphics{img/hand}
\caption{HÄR SKA AVSTÅNDSMÄTNINGARNA IN FRÅN MATLABPLOTTEN JOCKE HAR....}
\label{avstand}
\end{figure}

Figur~\ref{avstand} visar hur den matematiska modellen avviker från ett perfekt mätresultat för 10 olika. Det genom snittliga felet är XX procent och inom handens typiska arbetsområde (100-20 mm) är felet endast i genomsnitt XX provent med ett största fel på CPCPC mm. Om vi utgår från detta fel som största felmarginal kan vi med säkerhet särskilja objekt som ha yy mm differens på det klassificerande måttet.

Med denna väljas två olika testobjekt ut för att kontrollera om robothanden kan särskilja dem. det kan den och alla blir glada och dricker sprit och knullar. THE END
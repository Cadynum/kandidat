\chapter{Resultat}
I detta kapitel utvärderas robothandens prestanda i form av ett antal olika objekt den kan gripa och hur väl objektidentifieringen och tryckbegränsningen fungerar.
\section{Prestanda}
Efter utförda funktionstester har följande resultat uppnåtts. Robothanden kan:
\begin{itemize}
\item Gripa och lyfta ett mjölkpaket med en vikt av 1 kg. Se figur \ref{fig:milk}.
\item Lyfta upp en snusdosa. Se figur \ref{fig:snusgrepp}.
\item Gripa och lyfta en mutter av storlek M10 mellan tumme och pekfinger. Se figur \ref{fig:mutterlyft}.
\item Gripa cylinderformade objekt.Se figur \ref{fig:cylinder}.
\item Sluta sig till knuten näve från öppen hand på under en sekund.
\end{itemize}
\begin{figure}[H]
\begin{subfigure}[b]{0.4\textwidth}
                \centering
                \includegraphics[height=0.25\textheight]{img/milk}
                \caption{Verifiering lyft av mjölkpaket.}
                \label{fig:milk}
        \end{subfigure}~
                \begin{subfigure}[b]{0.4\textwidth}
                \centering
                \includegraphics[height=0.25\textheight]{img/snusgrepp}
                \caption{Verifiering lyft av snusdosa.}
                \label{fig:snusgrepp}
        \end{subfigure}


\begin{subfigure}[b]{0.4\textwidth}
                \centering
                \includegraphics[height=0.2\textheight]{img/mutterlyft}
                \caption{Verifiering lyft av mutter.}
                \label{fig:mutterlyft}
        \end{subfigure}~
        \begin{subfigure}[b]{0.4\textwidth}
                \centering
                \includegraphics[height=0.2\textheight]{img/cylinder}
                \caption{Verifiering lyft av cylinder.}
                \label{fig:cylinder}
        \end{subfigure}
\caption{Utförda grepp.}
\label{fig:greppagrap}
\end{figure}

\section{Objektidentifiering}

För att verifiera robothandens funktion kontrolleras objektidentifieringen, samt om robothanden kan begränsa trycket mellan robothandens fingrar och objektet då identifiering gjorts. Detta test ställer krav på den mekaniska konstruktionen, trådlösa överföringen, sensorerna i styrhandsken och robothanden samt att den matematiska modellen fungerar. Med andra ord behöver hela systemet fungera och samverka väl för att uppnå ett bra resultat.



För att kunna identifiera objekt behöver den matematiska modellen som beskriver hur robothandens fingrar är positionerade överensstämma med den faktiska konstruktionen av robothanden.
\begin{figure}[H]
\includegraphics{img/obj_id_matlab2}
\caption{Avståndet mellan det faktiska och beräknade värdet.}
\label{avstand}
\end{figure}

Figur~\ref{avstand} visar hur den matematiska modellen avviker från ett perfekt mätresultat för 18 olika mätningar. Modellens genomsnittliga avikelse är 20 procent medans inom handens typiska arbetsområde, som är \unit{40-100}{mm}, är felet endast i genomsnitt 11 procent med ett största fel på \unit{15}{mm}. Utgående från detta som största felmarginal kan robothanden med säkerhet särskilja objekt som har \unit{15}{mm} differens på det klassificerande måttet.

Utifrån detta väljs två objekt med måtten \unit{40}{mm} respektive \unit{100}{mm} för funktionstest som bekräftar att objektidentifieringen funkar.

\newpage
\section{Begränsning av tryckkraft}

För att verifiera att den tryckbegränsande funktionen utförs ett test där ett objekt identifieras och trycket begränsas.

\begin{figure}[H]
\includegraphics{img/masterplot}
\begin{tabular}{l l}
\textbf{\textcolor{red}{\line(1,0){25}}}
 & Önskat avstånd mellan fingerspetsarna\\
\textbf{\textcolor{orange}{\line(1,0){25}}}    & Faktiskt avstånd mellan fingerspetsarna\\
\textbf{\textcolor{blue}{\line(1,0){25}}}    & Kraft på fingerspets
\end{tabular}

\caption{Figuren visar ett händelseförlopp på 15 sekunder då rörelsen begränsas pga överträdelse av det maximalt tillåtna kraftvärdet för fri kontroll.}
\label{masterplot}
\end{figure}


Beskrivning av händelseförloppet:
\begin{table}[H]
\begin{tabularx}{\textwidth}{l X}
\textbf{0.0--6.0 s}    & Fri styrning av avstånd mellan fingerspetsarna eftersom trycksensorns värde är lägre än \unit{2}{N}.\\
\textbf{6.0--11.0 s} & Robothanden greppar objektet och kontakttrycket ökar tills gränsen överskrids. Maximal greppkraft för objektet identifieras via ett förutbestämt distansvärde mellan tummen och pekfingret. Robothanden låser servomotorernas vinklar för att förhindra ett starkare grepp.\\
\textbf{11.0--13.5 s} & Objektet glider ur greppet vilket resulterar i ett kontakttryck under gränsvärdet. Handen tillåts då rörelse fram tills trycket återfås och föregående procedur återupprepas.\\
\textbf{13.5--15.0 s}  & Användaren släpper objektet helt och kraften minskar momentant.
\label{tryckbegransning}
\end{tabularx}
\end{table}




%\subsubsection{Felkällor}
%\label{felkallor}
%Objektidentifieringen inehåller främst två felkällor som bidrar till en minskad precision. En av felkällorna är att modellen är baserat på önskade servolägen, dvs de lägen som specificerats av användaren. Om servomotorerna möter på motstånd som förhindrar rörelse till önskat läge kommer det faktiska avståndet och det avstånd beräknat genom modellen att skillja vilket bidrar till ett fel. Bild \ref{fig:objident1}-\ref{fig:objident2} illutrerar en situation då detta fel inträffar.
%
%\begin{figure}[H]
%
%        \begin{subfigure}[b]{0.5\textwidth}
%                \centering
%                \includegraphics{img/grepp}
%                \caption{Faktiskt servoläge}
%                \label{fig:objident1}
%        \end{subfigure}%
%       ~
%        \begin{subfigure}[b]{0.5\textwidth}
%
%                \includegraphics{img/grepp_verk}
%                \caption{Önskat servoläge}
%                \label{fig:objident2}
%        \end{subfigure}
%
%\end{figure}
%
%För att lösa detta problem behöver systemet få information om handens faktiska läge. Detta kan åstadkommas genom att ansluta vinkelgivare till fingrets leder alternativt till servomotorerna. \
%
%En annan stor felkälla är att handens konstruktion bidrar med låg precision, pga det mekaniska glappet. Mekanots låga toleranser gör att ett givet servoläge kan innebära många olika positioner. Felkällans magnitud ökar fort med ett ökat antal leder. För att öka precisionen behövs konstruktionen ändras för att exkludera det mekaniska glappet.

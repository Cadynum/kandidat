\section{Butterworthfilter}
\label{sec:butter}
För att filtera signaler används ett \emph{Butterworthfilter} av första ordningen. Filtret är ett lågpassfilter med minimal fasförskjutning för kausala lågpassfilter. Första ordningen används för att minimera fördröjningen, och för att undvika översläng. \citep{butterworth}
Låt $\omega_0$ vara brytfrekvensen i $\unit{rad/s}$ och $G(s)$ överföringsfunktionen i laplace-domän:
\begin{equation}
\label{eq:butter_s}
G(s) = \frac{\omega_0}{\omega_0 + s}
\end{equation}
Överföringsfunktionen i laplace-domän diskretiseras till Z-domän med en samplingsfrekvens på $f_s$ för att kunna implementera filtret på diskreta mätvärden:
\begin{equation}
\label{eq:butter_z}
G(z) = \frac{\omega_0 (z+1)}{2 f_s z-2 f_s + \omega_0 z + \omega_0}
\end{equation}
Genom att sätta $\omega_0 = 2\pi f_0$, där $f_0$ är brytfrekvensen i $\unit{Hz}$, och inverstransformera får vi följande differensekvation som kan användas för att i realtid implementera filtret digitalt:
\begin{equation}
\label{eq:butterdiff}
    y[n] = \frac{f_0(x[n]+x[n-1]) - y[n-1](f_0 - 2f_s)}
                {f_0 + 2f_s}
\end{equation}


\begin{figure}[ht]

\begin{subfigure}[t]{200pt}
\includegraphics{img/filter/butter_abs.pdf}
\caption{Förstärkning.}
\label{fig:butter_freq}
\end{subfigure}
~
\begin{subfigure}[t]{200pt}
\includegraphics{img/filter/butter_phase.pdf}
\caption{Fasförskjutning}
\label{fig:butter_phase}
\end{subfigure}

\caption{Frekvenssvar för ett första ordningens butterworthfilter med brytfrekvensen $f_0=15$. X-axeln är frekvens i Hz.}
\end{figure}


\subsection{Tidsfördröjning}
Butterworthfiltret fasförskjutning för olika frekvenser varierar (se figur~\ref{fig:butter_phase}).
För att mäta hur lång tidsfördröjning filtret introducerar för en viss frekvens $f$ kan vi normalisera fasförskjutningen i fourier-domänen mot dess periodtid , $\frac{1}{f}$. \\
Låt
\begin{itemize}
    \item $G(s)$ filtrets överföringsfunktion i laplace-domän, där vi studerar funktionens komplexa frekvens genom att sätta $s=2\pi f$. Se ekvation~\eqref{eq:butter_s}.
    \item $T_d(f)$ vara tidsfördröjningen i sekunder för frekvensen $\unit[f]{Hz}$
\end{itemize}

\begin{align}
    T_d(f) &= \frac{\arg\left(G(2\pi i f)\right)}{2\pi f}
\end{align}

Om vi vet de relevanta frekvenserna är i intervallet $\unit[0-f_0]{Hz}$ kan vi få fördröjningens medelvärde $\mu$ och standardavvikelse $\sigma$ genom att integrera $T_d(f)$:
\begin{align}
\label{eq:butterdelay}
\begin{split}
    \mu &= \frac{1}{f}\int_0^{f_0}{T_d(f) \, df} \\
    \sigma &= \sqrt{\frac{1}{f}\int_0^{f_0}{{\left(T_d(f)-\mu\right)^2 \, df}}}
\end{split}
\end{align}

Eftersom insignalerna inte har en uniform distribution av frekvenskomponenter i bandet 0--15 Hz ger ekvation~\eqref{eq:butterdelay} endast en approximering. Signaler med relativt många lågfrekventa komponenter, som \ref{fig:tryck_dft} och \ref{fig:dftflex}, kan ha en marginellt annorlunda fördröjning. För brytfrekvensen 15 Hz är dock det inbyggda felet så litet att det kan försummas.
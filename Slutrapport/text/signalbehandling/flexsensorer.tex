\section{Flexsensorer på styrhandsken}
Från flexsensorerna samplas en varierande analog spänning med en frekvens på 100 Hz. Signalen diskretireras sedan till heltal med en 10 bitars upplösning.
I det uppmätta mätvärdet finns det diverse störningar. Det är inte önskvärt att störningarna överförs till robothanden, eftersom servomotorerna i robothanden då konstant skulle ändra läge, vilket leder till en hackig upplevelse för användaren, slitage på servomotorerna, och extra strömförbrukning.

\begin{figure}[htb]
\includegraphics{img/filter/flex_raw.pdf}
\caption{De sex flexsensorerna uppmätta under 8.0s med en samplingsfrekvens på 100Hz när handen greppar ett objekt.}
\label{fig:rawflex}
\end{figure}

Genom att studera de sex signalerna i frekvensplanet kan 

\begin{figure}[htb]
\includegraphics{img/filter/flex_fourier.pdf}
\caption{Diskret fouriertransform av signalerna från figur~\ref{fig:rawflex}.}
\end{figure}


\begin{figure}[htb]
\includegraphics{img/filter/flex_filtered.pdf}
\caption{Signalerna från figur~\ref{fig:rawflex} filtrerade med ett Butterworth-filter av första ordningen med en brytningsfrekvens på $f=\unit[15]{Hz}$.}
\end{figure}


För att filtera signalen används ett \emph{Butterworth-filter} av första ordningen. Butterworth-filteret är ett lågpassfilter med små ripplar i passbandet. 

\begin{equation}
\label{eq:butter_s}
G(s) = \frac{\omega_0}{\omega_0 + s}
\end{equation}
Överföringsfunktionen överförs till Z-domän för att kunna implementera filtret digitalt.
\begin{equation}
\label{eq:butter_z}
G(z) = \frac{\omega_0 (z+1)}{2 f_s z-2 f_s + \omega_0 z + \omega_0}
\end{equation}
Genom att sätta $\omega_0 = f_0 \cdot 2\pi$ och inverstransformera får vi följande differensekvation:
\begin{equation}
    y[n] = \frac{f_0(x[n]+x[n-1]) - y[n-1](f_0 - 2f_s)}
                {f_0 + 2f_s}
\end{equation}


% breakfreq*((xp + x) - s->yp*(breakfreq - 2*hz))
% /(breakfreq+2*hz)

\section{Flexsensorer på styrhandsken}
Från flexsensorerna samplas en varierande analog spänning med en frekvens på 100 Hz. Signalen diskretireras sedan till heltal med en 10 bitars upplösning\footnote{Ett tal mellan 0--1023}.
I det uppmätta mätvärdet finns det diverse elektromagnetiska störningar. Det är inte önskvärt att störningarna överförs till robothanden, eftersom servomotorerna i robothanden då konstant skulle ändra läge, vilket leder till en hackig upplevelse för användaren, slitage på servomotorerna, och extra strömförbrukning.

\begin{figure}[htb]
\includegraphics{img/filter/flex_raw.pdf}
\caption{De sex flexsensorerna uppmätta under 8.0s med en samplingsfrekvens på 100Hz när en användare öppnar och stänger styrhandsken 4 gånger.}
\label{fig:rawflex}
\end{figure}

Genom att studera de sex flexsensorernas signaler i frekvensplanet (se figure~\ref{fig:dftflex}) ser vi att det finns en konstant störning, och att de önskvärda signalerna \emph{ungefär} är under 10--20Hz.
För att filtrera bort störningarna används ett \emph{butterworth-filter} (se \ref{sec:butter}). Brytfrekvensen valdes till $f_0=\unit[15]{Hz}$ efter flera tester där den upplevda fördröjningen av handens rörelser balanserades mot hur ``hackigt'' handen upplevdes röra dig. De filtrerade signalerna finns i figur~\ref{fig:filterflex}.


\begin{figure}[htb]
\includegraphics{img/filter/flex_fourier.pdf}
\caption{Diskret fouriertransform av signalerna från figur~\ref{fig:rawflex}.}
\label{fig:dftflex}
\end{figure}


\begin{figure}[htb]
\includegraphics{img/filter/flex_filtered.pdf}
\caption{Signalerna från figur~\ref{fig:rawflex} filtrerade med ett Butterworth-filter av första ordningen med en brytfrekvens på $f=\unit[15]{Hz}$.}
\label{fig:filterflex}
\end{figure}





% breakfreq*((xp + x) - s->yp*(breakfreq - 2*hz))
% /(breakfreq+2*hz)

\section{Trycksensorer på robothanden}

\begin{figure}[H]
\begin{tikzpicture}[node distance = 3.5cm, auto]
    \node [block] (sampling) {trycksensorernas spänning samplas};
    \node [block, right of=sampling] (filter) {störningar filtreras};
    \node [block, right of=filter] (norm1) {spänningen räknas om till tryck i Newton};

    \path [line] (sampling) -- (filter);
    \path [line] (filter) -- (norm1);
 \end{tikzpicture}
 \caption{Signalens väg för robothanden.}
 \end{figure}

\subsection{Sampling och filtrering}
\begin{figure}[h]
\includegraphics{img/filter/tryck_fourier.pdf}
\caption{Diskret fouriertransform av en trycksensors signal.}
\label{fig:tryck_dft}
\end{figure}
Från trycksensorerna samplas en varierande spänningen med en frekvens på 100 Hz.
Likt signalerna från flexsensorerna, se \ref{sec:flexsampling}, finns det diverse elektromagnetiska störningar i mätdatan som inte är önskvärda. Från figur~\ref{fig:tryck_dft} ser vi att majoriteten av de önskvärda frekvenserna är under 15Hz.
Filtreringen sker med ett första ordningens butterworthfilter, se \ref{sec:butter}, med $f_0=\unit[15]{Hz}$ som brytfrekvens.


\begin{figure}[ht]
\includegraphics{img/filter/tryck_dubbel.pdf}
\caption{Signal från en trycksensor före respektive efter filtrering. Den filtrerade signalen är förskjuten $+\unit[0.05]{V}$ i y-led för att belysa skillnaden. Signalen kan anta värden mellan $\unit[0-3.3]{V}$.}
\end{figure}



\subsection{Omvandling till Newton}
Samplad spänning relaterar inte linjärt till kraft.
Genom att mäta trycksensorns diskretirerade spänningsvärde vid flera punkter med känt applicerad kraft söker vi efter en funktion som beskriver mätdatan.
Funktionen observeras bete sig initialt avtagande exponentiellt ($1-e^{-x/b}$), för att sedan övergå i en $x^d$ kurva där $0<d<1$. Multiplicerat ger de två funktionerna ekvation~\eqref{eq:tryck}, där $b<0$ är den avtagande exponentiella kurvans lutning, $0<d<1$ potensfunktionens lutning och $a>0$ en skalningskonstant.
\begin{equation}
\label{eq:tryck}
    f(x) = a\left(1-e^{-x/b}\right) x^d
\end{equation}

\begin{figure}[H]
\includegraphics{img/tryck/before.pdf}
\caption{Uppmätta värden för kraftens relation till spänningen för pekfingrets trycksensor, och $f(x)$ med uppskattade $a,b,d$ konstanter för ekvation~\eqref{eq:tryck}.}
\label{fig:tryckbefore}
\end{figure}

Genom att invertera funktionen kan vi istället ge spänningen som insignal och få ut tryck. Funktionen $f^{-1}(x)$ har emellertid ingen sluten lösning. För att implementera funktionen effektivt används en tabell per trycksensor med i förväg numeriskt uträknade värden, där varje diskret spänningsvärde mellan 0--1023 ger en kraft.
\begin{equation}
\label{eq:tryckinv}
    f^{-1}[n] = y,\;\text{så att } n = f(y)
\end{equation}
Relationen mellan spänning och tryck är uppmätt upp till 10 N. Från sensorns datablad (se Appendix \ref{chp:komponentlista}) finns det referensgrafer. Mätdatan kan därför extrapoleras med relativt stor säkerhet upp till sensorns maxspecifikation, 20 N.

\begin{figure}[H]
\includegraphics{img/tryck/after.pdf}
\caption{Inverterad mätdata från figur~\ref{fig:tryckbefore} och $f^{-1}[n]$ (se ekvation~\ref{eq:tryckinv}) passad till pekfingrets trycksensor.}
\end{figure}

\begin{table}
\begin{tabularx}{350pt}{X r r r}
            & $a$ & $b$ & $d$ \\
Tumme       & 381 & 1.7 & 0.22 \\
Pekfinger   & 480 & 0.01 & 0.188 \\
Långfinger  & 285 & 1.42 & 0.305
\end{tabularx}
\caption{Passade konstanter för trycksensorerna. Se ekvation~\eqref{eq:tryck} och \eqref{eq:tryckinv}.}
\end{table}

\section{Trycksensorer på robothanden}

\begin{figure}[H]
\begin{tikzpicture}[node distance = 3.5cm, auto]
    \node [block] (sampling) {trycksensorernas spänning samplas};
    \node [block, right of=sampling] (filter) {störningar filtreras};
    \node [block, right of=filter] (norm1) {spänningen räknas om till tryck i Newton};

    \path [line] (sampling) -- (filter);
    \path [line] (filter) -- (norm1);
 \end{tikzpicture}
 \caption{Signalens från spänning till tryck i robothanden.}
 \end{figure}

\subsection{Sampling och filtrering}
\begin{figure}[h]
\includegraphics{img/filter/tryck_fourier.pdf}
\caption{Diskret fouriertransform av en trycksensors signal.}
\label{fig:tryck_dft}
\end{figure}
Från trycksensorerna samplas en varierande spänningen med en frekvens på 100 Hz.
Likt signalerna från flexsensorerna (se \ref{sec:flexsampling}) finns det diverse elektromagnetiska störningar i mätdatan som inte är önskvärda. Från figur~\ref{fig:tryck_dft} ser vi att majoriteten av de önskvärda frekvenserna är under 15Hz.
Filtreringen sker med ett första ordningens butterworth-filter (se \ref{sec:butter}) med $f_0=\unit[15]{Hz}$ som brytfrekvens.
\comment{todo: tidsfördröjning. autoreglering}



\begin{figure}[h]
\includegraphics{img/filter/tryck_dubbel.pdf}
\caption{Signal från en trycksensor före respektive efter filtrering. Den filtrerade signalen är förskjuten $+\unit[0.05]{V}$ i y-led för att belysa skillnaden. Signalen kan anta värden mellan $\unit[0-3.3]{V}$}
\end{figure}



\subsection{Omvandling till Newton}
Det spänningsvärde som samplas för varje trycksensor har inte ett linjärt samband mellan tryck och spänning och behöver därmed omvandlas.
Genom att mäta trycksensorns diskretirerade spänningsvärde vid flera punkter med känt applicerat tryck söker vi efter en funktion som beskriver mätdatan.
Funktionen observeras bete sig initialt avtagande exponentiellt ($1-e^{-x}$), för att sedan övergå i en $x^d$ kurva där $0<d<1$. Kombinerat ger de två funktionerna det oss ekvation~\eqref{eq:tryck}, där $a>0$ är en skalningskonstant.
\comment{skilj på tryck och kraft, tryck har enheten Pa = N/m^2 och kraft har N}

\begin{equation}
\label{eq:tryck}
    f(x) = a\left(1-e^{-x/b}\right) x^d
\end{equation}

\begin{figure}[H]
\begin{subfigure}[b]{0.49\textwidth}
\includegraphics{img/tryck/before.pdf}
\end{subfigure}
\begin{subfigure}[b]{0.49\textwidth}
\includegraphics{img/tryck/after.pdf}
\end{subfigure}
\end{figure}

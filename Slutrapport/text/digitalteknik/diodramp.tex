\section{Diodramp}
För återkoppling av trycket används en \emph{diodramp} med 8 led-lampor per finger på styrhandsken. För tre fingrar blir det totalt 24 dioder som ska kontrolleras individuellt.
Arduino Micro har inte 24 lediga utportar med sensorer och moduler inkopplade. Det skulle dessutom vara otympligt att dra 24 kablar mellan dioderna och mikrokontrollen.

Ett 8-bitars skiftregister (se \ref{chp:komponentlista}) används därför per diodramp. Ett skiftregistret har tre digitala inportar; data, klockpuls och asynkron reset, och 8 vippor \footnote{En enhet som sparar insignalen när klockan går från låg till hög.} som är kopplade till 8 utportar.
Data klockas in seriellt via dataporten och skiftar då varje bit ett steg. 

Antag att ett tillstånd $S_n$ med databitar $Q$ är
\[
    S_n = \{Q0, \dots, Q7\}
\]
Efter en positiv klockflank med det binära talet $IN$ på data-inporten kommer tillståndet vara
\[
    S_{n+1} = \{IN, Q0,\dots,Q6\}
\]
\comment{förklaringen slutar tvärt här. det säger mig ingenting att tillståndet försjuts eller vad som nu händer. vad händer sen, lyser det då eller?}
Från mikrokontrollen går en datapin till varje skiftregister, med en delad klocka och reset. Totalt behövs det 5 utportar med skiftregisterna.
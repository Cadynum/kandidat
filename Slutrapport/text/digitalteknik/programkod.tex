\section{Programkod}
Programkoden är strukturerad med flera filer som delas mellan robothanden och styrhandskens mikrokontroller.
Det finns även kod som inte direkt används i slutprodukten, men som är till för att testa enskilda funktioner på robothanden, som att mäta latenstiden för bluetooth-enheten.

All programkod finns på \emph{github}, genom \url{https://github.com/Cadynum/kandikod}. Använd följande kommando för att ladda ner en kopia av all programkod:
\begin{lstlisting}[basicstyle=\ttfamily]
git clone https://github.com/Cadynum/kandikod.git
\end{lstlisting}

\subsection*{Delad kod}
\begin{table}[H]
\begin{tabularx}{\textwidth}{p{3cm} X}
\textbf{angletodistance }    & Omvandlar servomotorernas vinklar till avstånd i mm. \\
\textbf{butterworth}         & Butterworthfiltrering av första ordningen effektivt implementerat med heltal. \\
\textbf{com }                & Hanterar kommunikation mellan bluetooth-enheter. Implementerad asynkront för att inte blockera resten av programmet när data väntas från bluetooth-enheten. \\
\textbf{state}               & Datastrukturer för tillstånd för tryck och flex-sensorer. Optimerar för att kunna serialiseras över bluetooth-enheten.
\end{tabularx}
\end{table}

\subsection*{Robothanden}
\begin{table}[H]
\begin{tabularx}{\textwidth}{p{3cm} X}
\textbf{robothand}       &
    Huvudprogrammet för robothanden. Samplar, filtrerar och omvandlar , återkopplar tryck och kommunicerar med robothanden. \\
\textbf{constants}       &
    Konstanter för programmet, som brytfrekvens för butterworthfiltret, och vilken periodtid mellan beräkningar. \\
\textbf{kraft}      &
    \emph{Lookup table} för att relatera en spänning mot ett tryck för varje finger. \\
\textbf{object}     &
    Räknar ut om handen ska begränsa trycket baserat på servovinklarnas börvärde.
\end{tabularx}
\end{table}


\subsection*{Styrhandsken}
\begin{table}[H]
\begin{tabularx}{\textwidth}{p{3cm} X}
\textbf{styrhandske} &
    Huvudprogrammet för styrhandsken.
    \begin{itemize}
        \item Samplar, filtrerar och normaliserar flexsensorer till servovinklar.
        \item Återkopplar tryck till användaren via diodramper.
        \item Kommunicerar med robothanden.
    \end{itemize} \\
\textbf{constants} & Som för robothanden, med andra värden.
\end{tabularx}
\end{table}

\subsection*{Hjälpprogram}

\section{Projektet ur miljö och hållbarhetsperspektiv}
Robothanden som byggts i detta projekt är som nämnts tidigare byggt främst i Meccano vilket är en återanvändbar byggsats. Då handen inte ska användas mer är det enkelt att montera isär den och använda delarna till annat. Vid händelsen  att Meccanot skulle bli skadat kan det återvinnas då stål är återvinningsbart. \cite{Worldsteel}
Ett bättre alternativ ur miljöperspektiv vore att först optimera konstruktion och val av material för att minska momentbehovet och därmed energiförbrukningen, och sedan köpa servon som drar mindre ström. Vid val av material bör även materialens miljöpåverkan beaktas för en så minimal belastning på miljön som möjligt. Mikrokontrollen kör bara 100 gånger per sekund för att spara energi men ändå inte ge för hög latenstid.

Utveckling och användning av robothänder i allmänhet har i ett större miljöperspektiv både för- och nackdelar. T.ex. minskas behovet av transport då en praktisk uppgift kan utföras via en fjärrstyrd robothand. Robothänder kommer även med stor sannolikhet vara en del av processen vid hantering av kärnavfall och hälsofarliga ämnen för att effektivt hantera dessa på ett bra sätt \ref{USLegal}.
Robothänder kan inte betraktas som miljöeffektiva förrän de negativa miljöeffekterna, från till exempel produktionen av handen och energiförbrukning vid användning, kan vägas upp av de effekter som undvikits genom att använda den. 

\chapter{Projektdel}
Robothanden har tagits fram i ett kandidatarbete under en termin. Då arbetstiden och budgeten är begränsade har avgränsningar gällande projektet gjorts. Detta kapitel syftar till att klargöra de projektmässiga avgränsningar som gjorts, föjt av en kort diskussion om hur dessa påverkat projektet samt om projektet i helhet.
%\section{Syfte}
 Syftet med projektet är att designa och konstruera en tryckkänslig robothand som via trådlös styrning klarar av att utföra stadiga grepp av olika förutbestämda föremål. Styrningen ska ske genom att operatören har på sig en handske vars rörelse robothanden ska följa. Vid ett visst kontakttryck mellan robothand och greppat objekt ska dock åtföljning av handskens rörelse upphöra och det är då istället greppkraften som styrs med handsken.\\
 Detta projekt breder ut sig över många ingenjörsmässiga ämnen och ett underliggande syfte för varje gruppmedlem är att delvis kunna fördjupa sina befintliga kunskaper på något område men även att utöka sin ingenjörsmässiga bredd och lära sig hur de olika tekniska områdena sammanflätas i en produkt.
 
 
\subsection{Avgränsingar}
För att begränsa arbetsbördan valdes att den mekaniska konstruktionen av handen skall utgå från Meccano. Detta för att minimera behovet av konstruktion av delar och snabbt komma igång med byggandet. Det antogs att Meccanot har tillräcklig styrka för att hålla för de krafter som skulle uppstå och därmed utesluts hållfasthetsberäkningar ur projektet.\\Vid objektidentifieringen som tillämpas i handen kommer vi endast att särskilja olika objekt med ett enda mått. Detta är en stor avgränsning jämfört med att mäta tre punkter relativt handen, som kan användas för att spänna upp en volym där möjlga objekt kan kontrolleras. Ett enda mått kan rymma en mängd olika objekt, men i tillämpningar kan ett mått ändå vara nog för att särskilja olika objekt beroende på variationen av möjliga objekt.

%\section{Mål}
Från början har följande mål satts upp och kommer diskuteras och utvärderas efter målen.
Handen:
\begin{itemize}
\item Ska kunna gripa och lyfta ett mjölkpaket med en vikt av ett kg. (Grasp 1 i Cutkoskys grepphierarki \ref{cutshand})
\item Ska kunna gripa och lyfta en mutter av storlek M10 mellan tumme och pekfinger. (Grasp 9 i Cutkoskys grepphierarki \ref{cutshand})
\item Ska klara att lyfta en last motsvarande ett kg på mitten av två fingrar.
\item Ska kunna lyfta upp en snusdosa.
\item Ska kunna reglera trycket som den applicerar på vissa förbestämda objekt.
\item Ska kunna identifiera förbestämda objekt genom att beräkna dess storlek utifrån avståndet mellan fingertopparna.
\item Styrning av handen ska vara intuitiv.
\item Maximal tid för handen att röra sig från maximalt öppen till en knuten näve är en sekund.
\end{itemize}
Målen verifieras genom att testa de olika momenten och dokumenteras med fotografering och videoinspelning. 

Projektet är begränsat till två läsperioder och har en budget på 5000 kronor. Även gruppmedlemmarnas kunskaper har inverkan.




Framtida rekomendedationer 

Slutsats
Meccano

sensorer går sönder
%\chapter{Resultat}
För att verifiera robothandens funktion kontrolleras objektidentifieringen, samt om robothanden kan begränsa trycket mellan robothandens fingrar och objektet då identifiering gjorts. Detta test ställer krav på den mekaniska konstruktionen, trådlösa överföringen, sensorerna i styrhandsken och robothanden samt att den matematiska modellen fungerar. Med andra ord behöver hela systemet fungera och samverka väl för att uppnå ett bra resultat.

Efter utförda funktionstester har följande resultat uppnåtts. Robothanden kan:
\begin{itemize}
\item Gripa och lyfta ett mjölkpaket med en vikt av ett kg. 
\item Gripa och lyfta en mutter av storlek M10 mellan tumme och pekfinger. 
\item Lyfta en last motsvarande ett kg på mitten av två fingrar.
\item Lyfta upp en snusdosa.
\item Sluta sig till knuten näve från öppen hand på under en sekund.
\end{itemize}

\subsection{Objektidentifiering}
För att kunna identifiera objekt behöver den matematiska modellen som beskriver hur robothandens fingrar är positionerade överensstämma med den faktiska konstruktionen av robothanden. 
\begin{figure}[H]
\includegraphics{img/obj_id_matlab2}
\caption{Avståndet mellan det faktiska och beräknade värdet.}
\label{avstand}
\end{figure}

Figur~\ref{avstand} visar hur den matematiska modellen avviker från ett perfekt mätresultat för 18 olika mätningar. Det genomsnittliga felet är 20 procent och inom handens typiska arbetsområde, som är \unit{40-200}{mm}, är felet endast i genomsnitt 11 procent med ett största fel på \unit{15}{mm}. Utgående från detta som största felmarginal kan robothanden med säkerhet särskilja objekt som har \unit{15}{mm} differens på det klassificerande måttet.

Med denna väljas två olika testobjekt ut för att kontrollera om robothanden kan särskilja dem. det kan den och alla blir glada och dricker sprit och knullar. THE END

\section{Diskussion}
En underliggande tanke med projektet var att robothanden kunde vara en grund för vidareutveckling i kommande kandidatarbeten. Då systemet i sin helhet innehåller så många ingenjörsmässiga områden, men tiden är begränsad, har fokus legat främst på att få till en fungerande helhet snarare än att specialutveckla eller optimera en viss del som exempelvis den mekansika konstruktionen eller objektidentifieringen. Det har antagits att Meccanot kommer hålla för de uppgifter robothanden ska utföra varför projektet avgränsade behovet av hållfasthetsberäkningar. Objektidentifieringen är begränsad till två dimensioner, alltså bara tummen och pekfingrets lägen är med i beräkningarna. 

Ett av de stora problemen med styrning av robothanden och dess precision vid identifiering av objekt är stabiliteten i kontruktionen. Tidigt i projektet togs beslutet att handen skulle byggas i meccano för att snabbt åstadkomma en prototyp och börja utveckla och testa styrningen. Att bygga i meccano resulterade i glapp mellan olika delar och töjningar av materialet som inte räknades med i den matematiska modellen. Detta gör att en korrekt bedömning av avståndet mellan fingertopparna för att identifiera objekt får låg precision varför antalet objekt som kan implementeras i programmet kraftigt begränsas (och målet att identifiera tre olika objekt inte uppnådes om detta ska va ett mål). Problem med skruvar som lossnar och delar som blir sneda gjorde dessutom att styrningsresultat kunde skilja sig från en gång till en annan. Detta innebär att en kalibrering av servomotorernas läge för greppning av ett bestämt objekt behövdes oftare än tänkt. Konstruktionen fungerar i övrigt bra. 

%Trots att det lagts mer tid på projektet än tänkt så har inte alla mål uppnåtts. Mycket tid i projektet har gått åt till att försöka få tag på personer på skolan som kan hjälpa till med att få tag på material som multimeter, resistanser, spänningsregulatorer m.m eller för att fråga om råd inom vissa områden där kandidatgruppens kompitens har varit bristande. Detta problem skulle kunna undvikas om material fanns tillgängligt i exempelvis ett lab där det även finns en ansvarig som kan vara till hjälp då det är problem med materialet. 
%Andra orsaker till förseningar har varit att olika saker går sönder. Exempelvis blev signalbehandlingsarbetet uppskjutet då många trycksensorer och flexsensorer gick sönder. Detta berodde till stor del på vår okunskap om deras ömtålighet. Problemet hade kunnat undvikas dels genom att vara försiktigare men också genom att i förväg köpa in reserver.

%Om ett liknande kandidatarbete ska göras i framtiden rekommenderas att en tydligare projektram tillhandahålls från början. Vi försökte under lång tid i början på projektet hålla många dörrar öppna och undvek att noga precisera målet med projektet.  Att sätta upp projektets mål och syfte visade sig vara mycket svårt då bedömningen av svårigheten i att uppnå vissa egenskaper eller tiden det tar att utföra uppgifter, grundas på kandidatgruppens kunskap. Målen har därför behövts alterneras allt eftersom. Man kan visserligen lära sig av detta problem men det tog mycket fokus från det tekniska arbetet. Ett mer välutstakat syfte med projektet skulle ge högre motivationsnivå.

 
 


 
\section{Projektet ur miljö och hållbarhetsperspektiv}
Robothanden som byggts i detta projekt är som nämnts tidigare byggt främst i Meccano vilket är en återanvändbar byggsats. Då handen inte ska användas mer är det enkelt att montera isär den och använda delarna till annat. Vid händelsen  att Meccanot skulle bli skadat kan det återvinnas då stål är återvinningsbart. \cite{Worldsteel}
Ett bättre alternativ ur miljöperspektiv vore att först optimera konstruktion och val av material för att minska momentbehovet och därmed energiförbrukningen, och sedan köpa servon som drar mindre ström. Vid val av material bör även materialens miljöpåverkan beaktas för en så minimal belastning på miljön som möjligt. Mikrokontrollen kör bara 100 gånger per sekund för att spara energi men ändå inte ge för hög latenstid.

Utveckling och användning av robothänder i allmänhet har i ett större miljöperspektiv både för- och nackdelar. T.ex. minskas behovet av transport då en praktisk uppgift kan utföras via en fjärrstyrd robothand. Robothänder kommer även med stor sannolikhet vara en del av processen vid hantering av kärnavfall och hälsofarliga ämnen för att effektivt hantera dessa på ett bra sätt \ref{USLegal}.
Robothänder kan inte betraktas som miljöeffektiva förrän de negativa miljöeffekterna, från till exempel produktionen av handen och energiförbrukning vid användning, kan vägas upp av de effekter som undvikits genom att använda den. 


\section{Diskussion}
En underliggande tanke med projektet var att robothanden ska vara en grund för vidareutveckling i kommande kandidatarbeten. Då systemet i sin helhet innehåller så många ingenjörsmässiga områden, men tiden är begränsad, har fokus legat främst på att få till en fungerande helhet snarare än att specialutveckla eller optimera en viss del som exempelvis den mekaniska konstruktionen eller objektidentifieringen. 




%Trots att det lagts mer tid på projektet än tänkt så har inte alla mål uppnåtts. Mycket tid i projektet har gått åt till att försöka få tag på personer på skolan som kan hjälpa till med att få tag på material som multimeter, resistanser, spänningsregulatorer m.m eller för att fråga om råd inom vissa områden där kandidatgruppens kompitens har varit bristande. Detta problem skulle kunna undvikas om material fanns tillgängligt i exempelvis ett lab där det även finns en ansvarig som kan vara till hjälp då det är problem med materialet. 
%Andra orsaker till förseningar har varit att olika saker går sönder. Exempelvis blev signalbehandlingsarbetet uppskjutet då många trycksensorer och flexsensorer gick sönder. Detta berodde till stor del på vår okunskap om deras ömtålighet. Problemet hade kunnat undvikas dels genom att vara försiktigare men också genom att i förväg köpa in reserver.

%Om ett liknande kandidatarbete ska göras i framtiden rekommenderas att en tydligare projektram tillhandahålls från början. Vi försökte under lång tid i början på projektet hålla många dörrar öppna och undvek att noga precisera målet med projektet.  Att sätta upp projektets mål och syfte visade sig vara mycket svårt då bedömningen av svårigheten i att uppnå vissa egenskaper eller tiden det tar att utföra uppgifter, grundas på kandidatgruppens kunskap. Målen har därför behövts alterneras allt eftersom. Man kan visserligen lära sig av detta problem men det tog mycket fokus från det tekniska arbetet. Ett mer välutstakat syfte med projektet skulle ge högre motivationsnivå.

 
 


 
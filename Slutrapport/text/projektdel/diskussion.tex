\section{Diskussion}
Här ska vi diskutera vad som gått fel och varit bra och typ rekommendera hur arbetet ska tas vidare på projekt nivå, tex klarare och tydligare mål då vi har haft det svårt med att det vart för öppet. tex en hand som ska kunna hålla det här, eller liknande. ``En del av nånting större...''

Ett av de stora problemen med styrning av robothanden och dess precision vid identifiering av objekt är stabiliteten i kontruktionen. Tidigt i projektet togs beslutet att handen skulle byggas i meccano för att snabbt åstadkomma en prototyp och börja utveckla och testa styrningen. Att bygga i meccano resulterade i glapp mellan olika delar och töjningar av materialet som inte räknades med i den matematiska modellen. Detta gör att en korrekt bedömning av avståndet mellan fingertopparna för att identifiera objekt får låg precision varför antlet objekt som kan implementeras i programmet kraftigt begränsas (och målet att identifiera tre olika objekt inte uppnådes om detta ska va ett mål). Problem med skruvar som lossnar och delar som blir snea gjorde dessutom att styrningsresultat kunde skilja sig från en gång till en annan. Detta innebär att en kalibrering av servomotorernas läge för greppning av ett bestämt objekt behövdes oftare än tänkt. Konstruktionen fungerar i övrigt mycket bra. 

Förseningar i konstruktion:Söndriga sensorer, dålig planering av beställningar. Dålig tillgång till material på skolan. Är det relevant?

början på en längre utveckling.
diskutera upplägget av kandidten. Inga tydlig mål. Ingen mening. Kunde delats upp på olika institutioner, mindre grupper. Otydligt syfte. 
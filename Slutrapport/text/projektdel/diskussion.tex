\section{Diskussion}
Här ska vi diskutera vad som gått fel och varit bra och typ rekommendera hur arbetet ska tas vidare på projekt nivå, tex klarare och tydligare mål då vi har haft det svårt med att det vart för öppet. tex en hand som ska kunna hålla det här, eller liknande. ``En del av nånting större...''

Vi bör kanske lägga in tidsplanen och diskutera utifrån den?

Ett av de stora problemen med styrning av robothanden och dess precision vid identifiering av objekt är stabiliteten i kontruktionen. Tidigt i projektet togs beslutet att handen skulle byggas i meccano för att snabbt åstadkomma en prototyp och börja utveckla och testa styrningen. Att bygga i meccano resulterade i glapp mellan olika delar och töjningar av materialet som inte räknades med i den matematiska modellen. Detta gör att en korrekt bedömning av avståndet mellan fingertopparna för att identifiera objekt får låg precision varför antalet objekt som kan implementeras i programmet kraftigt begränsas (och målet att identifiera tre olika objekt inte uppnådes om detta ska va ett mål). Problem med skruvar som lossnar och delar som blir sneda gjorde dessutom att styrningsresultat kunde skilja sig från en gång till en annan. Detta innebär att en kalibrering av servomotorernas läge för greppning av ett bestämt objekt behövdes oftare än tänkt. Konstruktionen fungerar i övrigt mycket bra. 

Trots att det lagts mer tid på projektet än tänkt så har inte alla mål uppnåtts. Mycket tid i projektet har gått åt till att försöka få tag på personer på skolan som kan hjälpa till med att få tag på material som multimeter, resistanser, spänningsregulatorer m.m eller för att fråga om råd inom vissa områden där kandidatgruppens kompitens har varit bristande. Detta problem skulle kunna undvikas om material fanns tillgängligt i exempelvis ett lab där det även finns en ansvarig som kan vara till hjälp då det är problem med materialet. 
Andra orsaker till förseningar har varit att olika saker går sönder. Exempelvis blev signalbehandlingsarbetet uppskjutet då många trycksensorer och flexsensorer gick sönder. Detta berodde till stor del på vår okunskap om deras ömtålighet. Problemet hade kunnat undvikas dels genom att vara försiktigare men också genom att i förväg köpa in reserver.

Om ett liknande kandidatarbete ska göras i framtiden rekommenderas att en tydligare projektram tillhandahålls från början. Att sätta upp projektets mål och syfte visade sig vara mycket svårt då bedömningen av svårigheten i att uppnå vissa egenskaper eller tiden det tar att utföra uppgifter grundas på kandidatgruppens kunskap. Målen har därför behövts alterneras allt eftersom. Man kan visserligen lära sig av detta problem men det tog mycket fokus från det tekniska arbetet. Ett mer välutstakat syfte med projektet skulle ge högre motivationsnivå.

 
En underliggande tanke med projektet var att robothanden kunde vara en grund för vidareutveckling. I kommande kandidatarbeten kan exempelvis något inrikta sig på optimering av konstruktionen och ett annat på objektidentifieringen. Båda dessa områden anses nämligen vara så stora att det skulle behövas. (Det finns liknande arbeten på andra skolor där utveckling av en av samma robothand sker kontinuerligt av nya elever varje år).

 
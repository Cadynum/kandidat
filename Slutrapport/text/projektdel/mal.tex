\section{Mål}
Från början har följande mål satts upp och kommer diskuteras och utvärderas efter målen.
Handen:
\begin{itemize}
\item Ska kunna gripa och lyfta ett mjölkpaket med en vikt av ett kg. (Grasp 1 i Cutkoskys grepphierarki \ref{cutshand})
\item Ska kunna gripa och lyfta en mutter av storlek M10 mellan tumme och pekfinger. (Grasp 9 i Cutkoskys grepphierarki \ref{cutshand})
\item Ska klara att lyfta en last motsvarande ett kg på mitten av två fingrar.
\item Ska kunna lyfta upp en snusdosa.
\item Ska kunna reglera trycket som den applicerar på vissa förbestämda objekt.
\item Ska kunna identifiera förbestämda objekt genom att beräkna dess storlek utifrån avståndet mellan fingertopparna.
\item Styrning av handen ska vara intuitiv.
\item Maximal tid för handen att röra sig från maximalt öppen till en knuten näve är en sekund.
\end{itemize}
Målen verifieras genom att testa de olika momenten och dokumenteras med fotografering och videoinspelning. 

Projektet är begränsat till två läsperioder och har en budget på 5000 kronor. Även gruppmedlemmarnas kunskaper har inverkan.




Framtida rekomendedationer 

Slutsats
Meccano

sensorer går sönder
\section{Robothandens principiella funktion}
HÄR SKA VI HA EN BRA ÖVERSIKTSBILD. FLOWCHART TYP
Robothanden kan delas in i X delsystem som samverkar för att uppfylla handens funktion. Delsystemen består av  BLA BLA VBAL

\section{Robothanden}
I detta avsnitt presenteras en överblick över de komponenter som utgör den mekaniska handen. Handen består av två fingrar och en tumme. På fingrar och tumme
\subsection{Handen}
För spara tid har Meccano™ använts som grund vilket har både fördelar och nackdelar jämfört med att bygga alla delar från början. Eftersom konstruktionen innehåller flera delar vars inbördes passform påverkar hur väl rörelser fungerar måste tillverkningen av delarna utföras med god nogrannhet. Meccanos standardbyggsats innehåller flera olika standardiserade delar som enkelt kan monteras i flera olika kombinationer.Delarna anses även ha tillräcklig nogrannhet för att slutprodukten ska kunna utföra de uppsatta målen. Eftersom konstruktionen av handen går snabbare med färdiga delar kan mer fokus läggas på den mer resurskrävande elektroniken. Nackdelen är att konstruktionen blir begränsad till att endast kunna tillverkas av tillgängliga komponeneter, men detta kringgås med konstruktion av enstaka kritiska komponenter.
BILD PÅ UTPROVNING AV GREPP
Handens utformning är en funktion av hur fingrarna önskas vara positionerade relativt varandra och detta utprovades i CAD-miljö utefter förmågan att utföra de önskade greppen. Handen har även tillräckligt stor yta för att möjliggöra integrering av aktuatorer, kontrollenhet och strömförsörjning i en enda enhet.



\subsection{Finger}

\subsection{Tumme}
\subsection{Fingertoppar och sensorer}
\section{Aktuering}
Kortfattande förklaring 
\subsection{Servon}
\subsection{Stag}
\subsection{Snören}

\section{Reglerhandsken}
Kortfattande Text om kommande


\section{Mikrokontroller}
Kortfattande Text om mikrokontroller
\section{Elektriska kretsar}
Kortfattande Teksti om 
\subsection{Strömförsörjning}
\subsection{Omkringliggande kretsar}
\section{Algoritmer}
Mer teksti


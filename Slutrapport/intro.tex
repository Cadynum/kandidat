\section{Inledning}

Robothänder kan användas som ett mycket effektivt hjälpmedel, inte bara för personer i behov av en protes, utan även i industriella tillämpningar. Monotona, fysiskt tunga, och direkt farliga arbeten kan utföras av en robothand istället för en människa.  Det pågår idag mycket forskning och utveckling av robothänder och detta har resulterat i mycket avancerade prototyper. en av de mest avancerade just nu är Shadow Dexterious Hand (TM) som med sina 20 aktuerade frihetsgrader och totalt 129 sensorer kan utföra det mesta som man kan begära av en mänsklig hand (ref längst ner). Stora hinder i utvecklingen har varit de höga kostnaderna, den komplexa fingerfärdigheten samt styrning. Kostnaden för en hand som kan efterlikna den mänskliga handens rörelser överstiger ofta 10 000 dollar (ref längst ner).
 

Möjligheten att fjärrstyra robothänder innebär att olika arbeten eller uppgifter kan utföras av en expert utan att experten finns på plats. Denna möjlighet kan vara direkt livräddande vid till exempel uppgiften att desarmera en bomb. Ett problem med att fjärrstyra en robothand är att operatören inte själv kan känna föremålet som hanteras vilket lätt kan leda till att ett för hårt eller löst grepp appliceras. Detta kan i sin tur resultera i att föremålet skadas eller i extremfallet, att en bomb detoneras. Vidare är det problematiskt att få styrningen att kännas intuitiv genom till exempel knapptryckningar.

%Föregående stycke är inte klart /Chris

\subsection{Syfte} %Eller snarare "Contributions".
Genom att utveckla en robothand som fjärrstyrs med hjälp av en styrhandske, där alltså robothanden skall röra sig som operatörens hand, blir styrningen intuitiv. Detta förutsätter att robothanden är designad för att likna den mänskliga handen i sina rörelser. För att undvika skador på objekt som robothanden hanterar ska ett bibliotek av objekt kunna implementeras i robothandens mikrokontroller som möjliggör objektidentifiering. Varje objekt har ett fördefinierat högsta tryck som robothanden får utsätta den för. Om operatören försöker greppa hårdare än det fördefinierade trycket ska robothanden reglera detta för att undvika skada.

Alltså en blandning mellan antropomorfisk robothand och greppverktyg...

För att uppnå detta krävs... mål (enligt Bengts råd)


+ hur vi ska tillföra något nytt 

\subsection{Rapportens Upplägg}
Rapporten är upplagd som följer..
I kapitel hmm till hmm ges en beskrivning av systemets design och funktion. Först diskuteras den mekatroniska designen av robothanden för att belysa dess rörelseomfång. I nästkommande kapitel behandlas styrhandsken med mikrokontroller och elektriska kretsar i detalj. Slutligen i kapitel hmm redogörs vilka algoritmer som utvecklats för att uppnå styrning och reglering.

I kapitel hmm finns resultat... efter dessa följer disskusion om bla bla och bla slutligen en slutsats där våra mål verifieras.





\subsection{Projektdefiniton} %Ta bort denna rubrik tror jag

\subsection{Avgränsningar}



\subsection{Vad inledningen ska innehålla enligt anvisningar}
Inledningen sätter in rapporten i ett sammanhang och visar dess
relevans och nyhetsvärde. Den fungerar som en introduktion till
hela rapporten och ska ge läsaren nödvändig information som
behövs för att ta del av dess innehåll.
Inledningen innehåller normalt en syftesformulering som ofta
ställs i relation till en bakgrund eller kort historik. I många fall
är syftesformuleringen nära relaterad till den
problemformulering som är viktig för att såväl läsare som
skribent ska kunna utnyttja rapporten väl. Det bör också stå
något om undersökningens eller experimentets omfattning och
anledningar till särskilda avgränsningar. Vidare bör också
metod finnas med, men endast i syfte att ange vilken typ av
undersökning som gjorts. Metoden utvecklas i andra avsnitt av
rapporten.
Man brukar ange bakgrund, syfte och metod som inledningens
obligatoriska funktioner. Ibland signaleras även centrala resultat
redan i inledningen.
Inledningen är den första sidan som sidnumreras.



http://www.shadowrobot.com/products/dexterous-hand/

http://www.nytimes.com/2013/03/30/science/making-robots-mimic-the-human-hand.html?_r=0


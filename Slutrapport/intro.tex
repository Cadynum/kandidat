
%Denna första del fungerar som bekgrund, men behöver ingen rubrik tycker jag. Hamnar direkt under rubriken Inledning.
Robothänder kan användas som ett mycket effektivt hjälpmedel, inte bara för personer i behov av en protes, utan även i industriella tillämpningar. Monotona, fysiskt tunga, och direkt farliga arbeten kan utföras av en robothand istället för en människa. I industritillämpningar kan robothänder vara förprogrammerade att utföra specifika uppgifter vilket inte kräver manuell styrning medan en protes kan styras med ett neuralt gränssnitt. Att manuellt kunna fjärrstyra robothänder har gett upphov till flera användningsområden. Möjligheten att finkänsligt kunna manipulera objekt utan att finnas på samma plats som objektet är användbart på många sätt. (Ska man rabbla exempel?) Ett problem med att fjärrstyra en robothand är att få styrningen att kännas intuitiv med till exempel knapptryckningar. En bra lösning på detta är att styrningen sker med en handske, så att robothanden efterliknar användarens rörelser. Detta förutsätter att robothanden har människoliknande rörelsemöjligheter. Ytterliggare ett problem är att undvika att skada eller felhantera föremålet med robothanden då användaren inte själv kan känna hur hårt den greppar.

 Det pågår idag mycket forskning och utveckling av robothänder och detta har resulterat i mycket avancerade prototyper.  En av de mest avancerade just nu inom genren som styrs med hjälp av en handske är Festo's ExoHand. Med 8 dubbelverkande pneumatiska aktuatorer och 16 trycksensorer kan den noga efterlikna användarens rörelse och dessutom ge haptisk återkoppling som möjliggör för användaren att noga reglera greppkraften (ref 3). (Skriv om force/velocity control handen! för att det är ett annat problematiskt sätt att lösa greppning av objekt.)En annan metod som utvecklats för att varsamt kunna hantera objekt med robothandprotes är force/velocity...

Stora hinder i utvecklingen av robothänder generellt har varit de höga kostnaderna, att efterlikna den mänskliga komplexa fingerfärdigheten samt styrning (ref 2). Kostnaden för en hand som kan efterlikna den mänskliga handens rörelser överstiger ofta 10 000 dollar (ref längst ner). Kostnaden stiger i regel med ökat antal kontrollerbara frihetsgrader och ökad komplexitet. 
 


%Möjligtvis ska det läggas ill lite om hur reglering gått till i tidigare modeller.
%Föregående stycke är inte klart /Chris

\section{Syfte}% "Contributions" -- Syfte.
I denna rapport redogörs för utvecklingen av ett system innehållande en robothand som fjärrstyrs med hjälp av en styrhandske. Varje finger och tumme har även 2 frihetsgrader vilket är mer än vad liknande projekt vanligtvis har. Robothanden konstrueras liknande en mänsklig hand i sina rörelser för att möjliggöra att styrningen är intuitiv. För att undvika skador på objekt som robothanden hanterar ska ett bibliotek av objekt implementeras i robothandens mikrokontroller som möjliggör objektidentifiering. Varje objekt har ett fördefinierat högsta tryck som robothanden får utsätta den för. Om operatören försöker greppa hårdare än det fördefinierade trycket ska robothanden reglera detta för att undvika skada. Genom att implementera denna reglering i robothandens mjukvara istället för till exempel mer avancerad haptisk återkoppling som kräver mekaniska funktioner i handsken hålls kostnaden nere.
Objektidentifieringen är baserad på igenkänning av objektets storlek genom att, utifrån vinklar på de styrande servomotorerna, beräkna avståndet mellan de tryckbelastade punkterna på robotfingrarna.
  Nedan visas ett schema för att representera systemets principiella funktion.

Systemets principiella funktion Bild!


\section{Prestanda}
Detta ska tydligen läggas i en "projektdel" i rapporten
Rapporten presenterar design av en robothand som uppfyller följande krav. Robothanden:
\begin{itemize}
\item Ska kunna gripa och lyfta ett mjölkpaket med en vikt av ett kg. (Grasp 1 i Cutkoskys grepphierarki \ref{cutshand}) %Glöm inte förklara Cutkosky nånstans!
\item Ska kunna gripa och lyfta en mutter av storlek M10 mellan tumme och pekfinger. (Grasp 9 i Cutkoskys grepphierarki \ref{cutshand})
\item Ska klara att lyfta en last motsvarande ett kg på mitten av två fingrar.
\item Ska kunna lyfta upp en snusdosa.
\item Styrning ska kunna anpassas till olika storlekar på styrande handen genom kalibrering.
\item En ovan användare skall efter kalibrering kunna utföra ovanstående mål. %(denna får vi nog ta bort tyvärr.)
\item Ska kunna reglera hur hårt den greppar tre stycken förbestämda objekt för att undvika skada.
\item Maximal tid för handen att sluta sig från maximalt öppen hand är en sekund.
\end{itemize}


\section{Avgränsningar}

ALLA avgränsningar, mål och syften ska INTE vara i den tekniska rapporten

 %Va passar i projektdel respektive teknisk rapport?
Budgeten för prototypen är 5000 kr vilket begränsar komplexiteten och antal frihetsgrader. Robothanden har därför endast 2 fingrar och en tumme med sammanlagt åtta frihetsgrader varav två är tvångsstyrda. Skelettet till robothanden har byggts i meccano (ref till förklaring av meccano) vilket begränsar designen till standardiserade mått och begränsar stabiliteten i kontruktionen. Robothanden är försedd med tre trycksensorer vilket avsevärt begränsar antal grepp som möjliggör objektidentifiering.

\section{Metod}
Genom litteraturstudier och studering av den egna handen har principen för den mekaniska designen framtagits. Vidareutveckling av detta skedde främst genom prövning, CAD-modellering och kraft- och momentberäkningar. Modellering av robothandens rörelse (rymdekvationer) har gjorts i MATLAB för att sedan förenklas och översättas till C++ och användas i mikrokontrollern Arduino... Får skriva detta ordentligt när vi närmar oss klarhetens tröskel.
 
"Vidare bör också metod finnas med, men endast i syfte att ange vilken typ av undersökning som gjorts. Metoden utvecklas i andra avsnitt av
rapporten."

\section{Rapportens Upplägg}
Rapporten är upplagd som följer..
I kapitel hmm till hmm ges en beskrivning av systemets design och funktion. Först diskuteras den mekatroniska designen av robothanden för att belysa dess rörelseomfång. I nästkommande kapitel behandlas styrhandsken med mikrokontroller och elektriska kretsar i detalj. Slutligen i kapitel hmm redogörs vilka algoritmer som utvecklats för att uppnå styrning och reglering.

I kapitel hmm finns resultat... efter dessa följer disskusion om bla bla och bla slutligen en slutsats där...






\section{Vad inledningen ska innehålla enligt anvisningar}
Inledningen sätter in rapporten i ett sammanhang och visar dess
relevans och nyhetsvärde. Den fungerar som en introduktion till
hela rapporten och ska ge läsaren nödvändig information som
behövs för att ta del av dess innehåll.
Inledningen innehåller normalt en syftesformulering som ofta
ställs i relation till en bakgrund eller kort historik. I många fall
är syftesformuleringen nära relaterad till den
problemformulering som är viktig för att såväl läsare som
skribent ska kunna utnyttja rapporten väl. Det bör också stå
något om undersökningens eller experimentets omfattning och
anledningar till särskilda avgränsningar. Vidare bör också
metod finnas med, men endast i syfte att ange vilken typ av
undersökning som gjorts. Metoden utvecklas i andra avsnitt av
rapporten.
Man brukar ange bakgrund, syfte och metod som inledningens
obligatoriska funktioner. Ibland signaleras även centrala resultat
redan i inledningen.
Inledningen är den första sidan som sidnumreras.



%1 http://www.shadowrobot.com/products/dexterous-hand/

%2 http://www.nytimes.com/2013/03/30/science/making-robots-mimic-the-human-hand.html?_r=0

%ExoHand 3 http://www.forbes.com/sites/singularity/2012/07/06/sophisticated-robotic-hand-also-doubles-as-a-human-exoskeleton/
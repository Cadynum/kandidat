\section{Rymdekvationer}

Som underlag till övriga beräkningar behövs en mattematisk modell över handens möjliga rörelsemönster. Bland annat behövs dessa beräkningar i ett senare stadie för att koppla önskad handposition till en motsvarande servovinkel.  Handens position bestäms av två inmatningsvariabler i form av vinklarna $\alpha$ och $\beta$. $\alpha$ utgår från y-axeln i ett koordinatsystem med origo i punkt P1 (se figur A.4.), medens $\beta$ utgår ifrån y-axeln i ett koordinatsystem med origo i punkt P3. Med andra ord kommer $\beta$ vinken att vara beroende av $\alpha$ då dess koordinat-system är kopplat till denna invariabel. Initialt behövs de vinklar som bestämmer samtliga punkters position beräknas vilket  görs med hjälp av ekvation \eqref{eq:rymdekv1} -- \eqref{eq:rymdekv9}. Ekvation \eqref{eq:rymdekv3} --\eqref{eq:rymdekv6} har tagits fram med hjäp av applicering av sinussattsen, medens \eqref{eq:rymdekv8} baseras på vinkelsummeringsregeln för en triangel. 

\begin{minipage}[t]{0.5\textwidth}
\begin{figure}[H]
\label{img:rymdimg1}
\includegraphics[width=1\textwidth]{img/rymdekv_length}
\caption{Definition av länger.}
\end{figure}
\end{minipage}
\begin{minipage}[t]{0.5\textwidth}
\begin{figure}[H]
\label{img:rymdimg2}
\includegraphics[width=0.93\textwidth]{img/rymdekv_ang}
\caption{Definition av vinklar}
\end{figure}
\end{minipage}


\medskip{}

\begin{equation}
\label{eq:rymdekv1}
L_{1}= L_{11}+L_{12}
\end{equation}

\begin{equation}
\label{eq:rymdekv2}
L_{2}= L_{21}+L_{22}
\end{equation}


\begin{align}
\label{eq:rymdekv3}
L_{11}=\frac{a\sin(\gamma)}{\sin(\theta)}  \\
\label{eq:rymdekv4}
L_{12}=\frac{b\sin(\lambda)}{\sin(\theta)} 
\end{align}


\begin{align}
\label{eq:rymdekv5}
L_{21}=\frac{a\sin(\xi)}{\sin(\theta)} \\
\label{eq:rymdekv6}
L_{12}=\frac{b\sin(\delta)}{\sin(\theta)} 
\end{align}

\begin{figure}[H]
\label{felbild}
\includegraphics[width=0.4\textwidth]{img/kandidat_figur_b}
\caption{Definition av $\gamma$}
\end{figure}

\begin{equation}
\label{eq:rymdekv7}
\gamma = \alpha + \pi - \beta 
\end{equation}

\begin{equation}
\label{eq:rymdekv8}
\xi+ \gamma+ \theta=\pi
\end{equation} 

\begin{equation}
\label{eq:rymdekv9}
\lambda+\delta+\theta=\pi 
\end{equation} 

Ur den mattematiska modellen erhålls nio okända variabler, fem okända vinklar och 4 okända längder vilka löses med ekvation \eqref{eq:rymdekv1} -- \eqref{eq:rymdekv9}. Sammanställningen av dessa ekvationer resulterar i ekvation \eqref{eq:rymdekv10} -- \eqref{eq:rymdekv11}. Då vinklarna $\theta$ och $\lambda$ är beroende av varandra går dessa inte att lösa analytiskt. För att beräkna $\theta$ och $\lambda$ applicerades en iterativ lösningsmodell i matlab. 
Genom att initiera med en approximation av $\theta$ kan ett $\lambda$ räknas ut ekvation \eqref{eq:rymdekv11} vilket genererar ett nytt $\theta$ med hjälp av ekvation \eqref{eq:rymdekv10}. Detta ittereras tills en önskad tolerans har uppnåtts mellan det gamla $\theta$-värdet och det nya. Den antagna itterationsmodellen är stabil med snabb konvergens då problemets komplexitet är relativt låg.




\begin{align}
\label{eq:rymdekv10}
\theta{} &= \arcsin\left(\frac{a\sin(\pi-\theta-\gamma)+b\sin(\pi-\theta-\lambda )}{L_{2}}\right) \\
\bigskip{}
\label{eq:rymdekv11}
\lambda{} &=\arcsin\left(\frac{L_{1}\sin(\theta)-a\sin(\gamma)}{b}\right) 
\end{align}

\[
\begin{cases}
L_{0}=\unit[59]{mm} \\
L_{1}=\unit[49]{mm}\\
L_{2}=\unit[49]{mm}\\
L_{3}=\unit[26]{mm}\\
a=\unit[12]{mm} \\
b=\unit[12]{mm} \\
\end{cases}
\]

När vinkelitterationen genomförts kan koordinaterna för samtliga punkter hittas med hjälp av den beräknade vinkeldatan. 


\begin{table}[H]
    \begin{tabular}{|c|l|l|}
        \hline
        Punkt  & X-Koordinat                          & Y-Koordinat                          \\ \hline
        1      & 0                                    & 0                                    \\ \hline
        2      & $(L_{o}-a)\sin(\alpha)$              & $(L_{o}-a)\cos(\alpha)$              \\ \hline
        3      & $L_{o}\sin(\alpha)$                  & $L_{o}\cos(\alpha)$                  \\ \hline
        5      & $L_{o}\sin(\alpha)+L_{2}\sin(\beta)$ & $L_{o}\cos(\alpha)+L_{2}\cos(\beta)$ \\
        \hline
    \end{tabular}
\end{table}

Vinkeln mellan staget ($L_{2}$), och en axel parallel med x-axeln utgående från punkt p2 benäms som ($\eta$) och beräknas enligt ekvation \eqref{eq:rymdekv12}. $\eta$ kan sedan användas för att räkna ut punkt P4 enligt ekvation \eqref{eq:rymdekv13}.

\begin{equation}
\label{eq:rymdekv12}
\eta=\frac{\pi}{2}-\alpha-\xi
\end{equation}

\begin{equation}
\label{eq:rymdekv13}
P4=[P2_{x}+L_{1}\cos(\eta),P2_{y}+L_{1}\sin(\eta)]
\end{equation}

För att räkna ut fingerspettsens koordinater behövs ytterliggare en vinkel introduceras. Denna vinkel benäms som $\omega$ och defineras enligt figur A.7 samt ekvation \ref{eq:rymdekv14}.

\begin{figure}[H]
\label{rymdekv_img4}
\includegraphics[width=0.4\textwidth]{img/rymdekv_omega}
\caption{Definition av $\omega$}
\end{figure}

\begin{equation}
\label{eq:rymdekv14}
\omega=\delta-(\frac{\pi}{2}-\eta)
\end{equation}

\begin{equation}
\label{eq:rymdekv15}
P6=[P4_{x}+(b+L_{3})\sin(\omega),P4_{y}+(b+L_{3})\cos(\omega)]
\end{equation}

Slutgiltligen används ekvation \eqref{eq:rymdekv15} för att bestämma koordinaterna för punkt P6. Med hjälp av denna mattematiska modell kan alltså fingrets position bestämmes utifrån två input vinklar $\alpha$ och $\beta$. Den mattimatiska modellen har använts för att i matlab göra simuleringar av fingrets rörelse som funktion av varierande $\alpha$ och $\beta$. 
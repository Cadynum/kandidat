''We recommend that you have one part in the report where you describe project related issues. There you can describe the purpose, goals, limitations and constraints. This part should be a clearly separated from the rest of the project which is written as a technical report with focus on information for the reader. Also, the goals you have but you never reached can be described in the project related part.'' - Jonas


\section{Syfte}
 Syftet med projektet är att designa och konstruera en tryckkänslig robothand som via trådlös styrning klarar av att utföra stadiga grepp av olika förutbestämda föremål. Styrningen ska ske genom att operatören har på sig en handske vars rörelse robothanden ska följa. Vid ett visst kontakttryck mellan robothand och greppat objekt ska dock åtföljning av handskens rörelse upphöra och det är då istället greppkraften som styrs med handsken.
 
\section{Mål}
Från början har följande mål satts upp och kommer diskuteras och utvärderas efter målen.
Handen:
\begin{itemize}
\item Ska kunna gripa och lyfta ett mjölkpaket med en vikt av ett kg. (Grasp 1 i Cutkoskys grepphierarki \ref{cutshand})
\item Ska kunna gripa och lyfta en mutter av storlek M10 mellan tumme och pekfinger. (Grasp 9 i Cutkoskys grepphierarki \ref{cutshand})
\item Ska klara att lyfta en last motsvarande ett kg på mitten av två fingrar.
\item Ska kunna lyfta upp en snusdosa.
\item Fingerspetsen ska kunna inta två olika lägen utan att flytta handen. (trycka på två olika knappar)
\item Information om trycket som handen påverkar objektet med ska mätas.
\item Handen skall kunna kontrollera trycket som den applicerar på objekt.
\item En ovan användare skall efter kalibrering kunna utföra ovanstående mål.
\item Maximal tid för handen att röra sig från maximalt öppen till en knuten näve är en sekund.
\end{itemize}
Målen verifieras genom att testa de olika momenten och dokumenteras med videoinspelning. 

Projektet är begränsat till två läsperioder och har en budget på 5000 kronor vilket påverkar slutresultatet. Även gruppens kunskaper har inverkan.




Framtida rekomendedationer 

Slutsats
Meccano

sensorer går sönder

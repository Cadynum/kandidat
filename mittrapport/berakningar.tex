\section{Beräkningar}
För att kunna dimensionera servona och få en uppfattning om handens styrka behövdes de krafter som påverkar handen räknas ut. Beräkningarna gjordes genom att frilägga ett enskilt finger och rita ut de moment och krafter som verkar på fingret. Då gripdonet kommer att ha möjligheten av inta flera olika lägen med varierande kraftriktningar skapades två greppfall för att underlätta beräkningar. De olika greppfallen består av vertikal last och omfattning av objekt. Greppfallen bygger delvis på de uppsatta målen och delvis på rörelser som ger en bra uppfattning om den totala styrkan.

\paragraph{Tröghetsmoment}

En uppskattning av tröghetsmomentet räknas fram för att se om den kommer påverka val av servomotor. För att säkerställa kravet att handen kan gå från öppet till stängt läge på mindre än en sekund beräknades tröghetsmomentet som krävs i led 1 för att röra fingret från läge 1 till läge 2 enligt figur 4.4, på en sekund. I beräkningen betraktas fingret som ett stelt rätblock som roterar kring z-axeln. Tröghetsmomentet blir nämligen mindre då led 2 och 3 böjs, genom att bortse från denna böjning fås en säkerhetsmarginal.
 
 \begin{figure}[H]
\includegraphics[width=0.5\textwidth]{img/troghetsmoment}
\caption{Beräkning av tröghetsmoment.}
\end{figure}

$$ a=0.150\unit{m}$$ $$ b=0.013 \unit{m};  m=0.100\unit{kg} $$
$$ I_x=\frac{1}{12}\cdot m\cdot b^2+\frac{1}{3}\cdot m\cdot c^2 $$

$$ I_x\cdot \dot{\omega}=M_1 $$

Vi ser här att bidraget till momentet från tröghetsmomentet är litet och kommer därför att bortse från dess påverkan härefter. 

 
  \paragraph{Krokgrepp}
   Vid vertikal last belastas fingret med en punktkraft motsvarande 1 kg på mitten av ben 2. Den yttre fingerleden tas därmed inte med i beräkningarna eftersom den i första hand ska utföra finmotoriska uppgifter där kraven på greppstyrka inte är lika viktiga. 
Figur 4.5 visar ett godtyckligt läge för ben 1 och 2. Eftersom den givna lasten är relativt hög försummas strukturens egentyngd tillsammans med friktion och fjädrarnas återförande kraft. Denna försumning kompenseras dock av en säkerhetsfaktor på 1,7.

\begin{figure}[H]
\includegraphics[width=0.5\textwidth]{img/teckning}
\caption{Krokgrepp.}
\end{figure}

Utifrån nedanstående ekvationer kunde en rörelse simuleras i matlab som visade hur kraften förändras med fingrets position. Då Krokgreppet innebär att två fingrar kommer bära lasten så delas kraften F på 2.
$$F=\frac{9.82}{2} [N]$$
$$l_a=l_1*sin(\pi-\alpha_1)$$
$$\theta=\alpha_2-\frac{3\pi}{2}+\alpha_1$$
$$l_b=\frac{l_2}{2}cos(\theta)$$
$$M_2=F*\frac{l_b}{2}$$
$$M_1=F*(l_b+l_a)$$

\begin{figure}[H]
\includegraphics[width=\textwidth]{img/fingermoment}
\caption{Visar hur momentet förändras i varje led efter en bestämd rörelse enligt $\frac{\pi}{2} <\alpha_1<\pi  $ och $\frac{\pi}{2} <\alpha_2<\pi$.}
\end{figure}



Diagramemt visar momentet som krävs i lederna vid 30 olika positioner där vinklarna $\alpha_1$ och $\alpha_2$ har förändrats med samma vinkel varje position. Från diagrammet framgår att högsta momentet uppstår i led 1 och är 0.553 [Nm]. För att vara på säkra sidan gjordes samma beräkning fast $\alpha_2$ hölls på 180 grader. Då uppstod som väntat största momentet i led 1 då fingret hölls rakt i horisontellt läge. Detta moment uppgick till 0.639 [Nm].

\paragraph{Omfattning av objekt}
Vid omfattning av objekt ska ett objekt i storleksordning av ett mjölkpaket gripas med tillräcklig kraft för att lyfta upp det. Friktionskoefficienten mellan kontaktytor är 0,65 vilket är en vanlig koefficient mellan stål och plast.
 
    
För att beskriva fingrets läge i rymden och få en uppfattning om området som fingerspetsen kommer att röra sig i beskrevs fingret med matematiska formler. Då fingerdel nummer ett och tre är sammanlänkade med ett stag kommer fingerspetsens läge endast vara beroende av två vinklar och de olika delarnas mått.



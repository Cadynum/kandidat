\section{Syfte}
Syftet med detta projekt är att skapa en robothand som ska utgöra en tidig grund för fortsatt vidareutvecklingsarbete. För att leda arbetet har ett flertal mål med avseende på gripfunktionalitet och tekniska egenskaper skapats som i slutet av projektet kontrolleras och utvärderas. En skillnad mellan det aktuella projektet och tidigare projekt (Lägg in referenser till olika projekt) är fingrets möjlighet att röra sig i olika lägen. Varje finger ska vara ledad i tre punkter och ha två frihetsgrader, dvs. den kommer vara tvångsstyrd i den sista leden.
 Till skillnad från liknande tidigare projekt som innehåller specialtillverkade delar, kommer det aktuella projektet till största  delen konstrueras med hjälp av färdigtillverkade standarddelar och vanliga modellservon. Detta görs för att undersöka hur enkelt det är att tillverka ett funktionsdugligt gripdon till ett låg pris. Då färdiga standarddelar används ska projektet resultera i en färdig hand som ger ett bättre helhetsperspektiv över handens utformning än ett välutvecklat finger.
 Robothanden ska styras med en handske som operatören använder för att reglera handens rörelse. Genom att använda en handske blir inmatningen naturlig vilket gör det lätt och intuitivt att kontrollera hande. Signalerna mellan handsken och robothanden skickas trådlöst för att möjliggöra flexibelt användande. För att inte förstöra objektet som grips kommer det finnas trycksensorer, så kallade FSR-sensorer, som mäter trycket och ger visuell återkoppling till användaren.
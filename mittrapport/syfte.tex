\section{Syfte}
Syftet med detta projekt är att skapa en robothand som ska utgöra en tidig grund för fortsatt vidareutvecklingsarbete. För att leda arbetet har ett flertal mål med avseende på gripfunktionalitet och tekniska egenskaper skapats som i slutet av projektet kontrolleras och utvärderas. En skillnad mellan det aktuella projektet och tidigare projekt (Lägg in referenser till olika projekt) är fingrets möjlighet att röra sig i olika lägen. Varje finger ska vara ledad i tre punkter och ha två frihetsgrader, dvs. den kommer vara tvångsstyrd i den sista leden.
 Till skillnad från liknande tidigare projekt som innehåller specialtillverkade delar, kommer det aktuella projektet till största  delen konstrueras med hjälp av färdigtillverkade standarddelar och vanliga modellservon. Detta görs för att undersöka hur enkelt det är att tillverka ett funktionsdugligt gripdon till ett låg pris. Då färdiga standarddelar används ska projektet resultera i en färdig hand som ger ett bättre helhetsperspektiv över handens utformning än ett välutvecklat finger.
 Robothanden ska styras med en handske som operatören använder för att reglera handens rörelse. Genom att använda en handske blir inmatningen naturlig vilket gör det lätt och intuitivt att kontrollera handen. Signalerna mellan handsken och robothanden skickas trådlöst för att möjliggöra flexibelt användande. För att inte förstöra objektet som grips kommer det finnas trycksensorer, så kallade FSR-sensorer, som mäter trycket och ger visuell återkoppling till användaren.\\
 
 
 \textbf{Syfte 2}
 Syftet med projektet är att designa och konstruera en tryckkänslig robothand som via trådlös styrning klarar av att utföra stadiga grepp av olika förutbestämda föremål. Styrningen ska ske genom att operatören har på sig en handske vars rörelse robothanden ska följa. Vid ett visst kontakttryck mellan robothand och greppat objekt ska dock åtföljning av handskens rörelse upphöra och det är då istället greppkraften som styrs med handsken.\\
 Detta projekt breder ut sig över många ingenjörsmässiga ämnen och ett underliggande syfte för varje gruppmedlem är att delvis kunna fördjupa sina befintliga kunskaper på något område men även att utöka sin ingenjörsmässiga bredd och lära sig hur de olika tekniska områdena sammanflätas i en produkt.
 
 
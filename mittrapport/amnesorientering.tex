\section{Ämnesorientering}
Det pågår idag mycket forskning och utveckling av robothänder. En indelning av området kan göras mellan händer designade för att efterlikna den mänskliga anatomin och fysiologin, antropomorfiska, samt händer designade för praktiska ändamål, greppverktyg. \cite{Bicchi} 1 i listan]

Att efterlikna den mänskliga handen är en tekniskt komplicerad uppgift med hänsyn till flexibilitet, styrka, uthållighet, finmotorik samt sensorisk förmåga. Inom den antropomorfiska grenen finns ett intressant tillämpningsområde som proteser. Inom detta fältet finns idag finns idag allt från simpla, rent mekaniska [ ref 2]  till mekatroniskt avancerade som t.ex. SmartHand  [ref 3 ] med 16 frihetsgrader aktuerade av fyra motorer, utrustad med 40 sensorer och med ett gränssnitt utvecklat för att möjliggöra styrning via kroppens egna nervsystem.

Bland andra antropomorfiska robothänder som inte har protesernas krav på vikt och mänsklig passform finns bl.a. Shadow Dextreous Hand [ref 4], med 24 frihetsgrader varav 20 kan aktueras oberoende av varandra och en total vikt på 4.2 kg, som mycket nära kan imitera rörelseomfånget hos den mänskliga handen, samt Robonaut 2 Hand [ref 5] utvecklad av Nasa och General Motors för att kunna använda verktyg anpassade för människor.

Greppverktygen är ett mycket brett fält där robothänderna är designade efter specifika kravbilder utifrån respektive tillämpningsområde, snarare än kravet människolikhet. Detta gör att en gräns mellan vad som kan klassas som en robothand och vad som bör klassas som ett automatiserat gripdon är svår att dra. Om kravet mångsidighet används som avdelare inkluderas även projekt som ‘Universal robotic gripper based on the jamming of granular material’ [ref 6], vilket fungerar efter principen att en påse med löst packat kaffepulver pressas mot ett objekt så att det formar sig efter objektets gränssnitt, därefter appliceras ett undertryck i påsen vilket gör att kaffepulvret och påsen stelnar kring objektet och på så vis är objektet greppat.
Därför har kandidatgruppen valt att definiera robothänder som de greppverktyg som har ledade extremiteter, fingrar, som utgår från en gemensam baskonstrukton, handen, och därmed exkluderas även olika former av gripklor.
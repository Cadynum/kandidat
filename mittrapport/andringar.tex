\section{Uppföljning av planeringsrapporten}
När projektet befann sig i ett tidigare stadie hölls det abstrakt för att vara öppet för så många olika lösningar som möjligt. Arbetet var anpassat mer för en produktutvecklingsprocess. När arbetet inleddes ändrades karaktären till att vara mer ett målbestämt tekniskt arbete med tydligare detaljer om vad som ska göras och hur. Mer tyngd lades även på att skapa ett mindre antal mycket klara mål istället för att ha ett större antal mindre definerade mål.

De mekatroniska delmålen avklarades snabbare än mjukvarudelmålen eftersom resurserna för mekaniken var tillgängliga tidigare. Framtagningen av den mekaniska  designen gick snabbare än planerat på grund av möjligheten att experimentera med existerande meccanodelar utan att behöva motivera inköp med beräkningar.

Enligt tidsrapporten skulle stort fokus läggas på framtagning av de olika koncepten. När processen väl började framgick det snabbt att ett koncept visade sig vara överlägset alla andra. Det gjorde att ett konceptval snabbt kunde göras och att prototypbygge och  vidareutvecklingen startade tigigare. Utvecklingsarbetet av detaljerna visade sig ta längre tid eftersom många delars utformning påverkade den slutgiltiga produktens funktion. För att den skulle klara att utföra de uppsatta målen krävdes att delarna utformades noga 

Det här är saxat från Gdocs och behöver väl ordnas
Uppföljning av projektplan:


Se till att tidsplanen innehåller klara leveranser och deadlines...:
För att göra detta krävs att tidiagre punkter är klara.

SammanfattningEnligt planeringsrapportens tidsplan skulle vi påbörja CAD modellering i arbetsvecka 9, beräkning och simulering i vecka 10 och prototypbygge i vecka 11. Alla dessa tre var redan påbörjade i vecka 4 dock. Detta beror på att vår bild av hur  projektet ska utföras har förändrats sedan vi gjorde planeringsrapporten.  När projektet befann sig i ett tidigare stadie hölls det abstrakt för att vara öppet för så många olika lösningar som möjligt. Arbetet var anpassat mer för en produktutvecklingsprocess. När arbetet inleddes ändrades dock karaktären till att vara mer av ett målbestämt tekniskt arbete med tydligare riktlinjer för vad som ska göras. Mer tyngd lades alltså på att snabbt skapa ett mindre antal klara mål för att kunna påbörja det tekniska arbetet snabbare.

De mekatroniska delmålen avklarades snabbare än mjukvarudelmålen eftersom resurserna för mekaniken var tillgängliga tidigare. Framtagningen av den mekaniska  designen gick snabbare än planerat på grund av valet att bygga prototypen i meccano. Då fanns möjligheten att experimentera med existerande meccanodelar utan att behöva motivera inköp med beräkningar. Dessutom blev designen begränsad till meccanos mått.

Enligt tidsplanen skulle de första 7 veckorna läggas på produktutvecklingsprocessen. Det förtydligades för oss senare att detta inte var vad institutionen där vi gör vårt kandidatarbete var intresserad av. Dessutom framgick det snabbt att ett koncept som vi tagit fram visade sig vara överlägset alla andra. Det gjorde att ett konceptval snabbt kunde göras och eftersom valet inte behövde redovisas så noga som vi först planerat kunde andra uppgifter påbörjas tidigare.
Vi ligger alltså väldigt bra till i vår projektplan från planeringsrapporten, men detta är egentligen inte längre relevant.



Uppdatering av projektplan för den kvarvarande delen, kommentera ändringar.

Projektplanen har nu fått mycket klarare upplägg än tidigare med klarare delmål. Genom att betrakta robothanden och styrning av den som ett system och sedan bryta ner detta system i mindre delar på ett tydligare sätt än tidigare blev planeringen också lättare att göra.
Det som kvarstår är att planera och konstruera en fungerande 


Styrning:
Vi skulle kunna ha funktionen med sensorerna på robothanden att vid första beröring så så stannar robothanden den delen som berör tills alla eller åtminstone 1 finger och tummen berör.
Man skulle ha en knapp som ökar trycket stegvis för att styrka greppet eftersom manuella styrningen med handsken då robothanden håller i nåt blir konstig. I så fall skulle även trycksensorernas utslag få en tydlig användning så man kan se hur hårt greppet blir vid varje knapptryckning.

Vi skulle kunna ha en knapp som aktiverar automatiska regleringen av trycket som vi sagt att vi ska ha (om vi nu ska ha det) så att om man vill, kan man klämma sönder saker.  Dessa två knappar känns som de borde vara enkla att tillämpa.

Hur mäts handskens rörelse? Detta måste man veta för att kunna veta vad man har för styrsignal i Simulinken.


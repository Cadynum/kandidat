\section{Uppföljning av  planeringsrapport}

Enligt planeringsrapportens tidsplan skulle CAD modellering utföras i arbetsvecka 9, beräkning och simulering i vecka 10 och prototypbygge i vecka 11. Alla dessa tre var redan påbörjade i vecka 4 dock. Detta beror på att bilden av hur projektet ska utföras har förändrats sedan planeringsrapporten.  När projektet befann sig i ett tidigare stadie hölls det abstrakt för att vara öppet för så många olika lösningar som möjligt. Arbetet var anpassat mer för en produktutvecklingsprocess.

När arbetet inleddes ändrades dock karaktären till att vara mer av ett målbestämt tekniskt arbete med tydligare riktlinjer för vad som ska göras. Mer tyngd lades alltså på att snabbt skapa ett mindre antal mycket klara mål för att kunna påbörja det tekniska arbetet snabbare.

Arbetet med de mekatroniska delmålen kunde påbörjas snabbare än mjukvarudelmålen eftersom resurserna för mekaniken var tillgängliga tidigare. Framtagningen av den mekaniska  designen gick snabbare än planerat på grund av valet att basera prototypen på Meccano. Då fanns möjligheten att experimentera med existerande Meccanodelar utan att behöva motivera inköp med beräkningar. Dessutom blev de möjliga designvalen begränsade till att vara genomförbara med Meccano.

Enligt tidsplanen skulle de första 7 veckorna läggas på produktutvecklingsprocessen. Det framgick dock snabbt att ett koncept som tagits fram visade sig vara överlägset alla andra. Det gjorde att ett konceptval snabbt kunde göras och att en vidareutveckling kunde börja tidigare än planerat.

Projektplanen har nu fått mycket klarare upplägg än tidigare med klarare delmål. Genom att betrakta robothanden och styrning av den som ett system och sedan bryta ner detta system i delsystem med respektive delfunktioner blev planeringen också lättare att göra och tydliga leveranser kan sättas upp.












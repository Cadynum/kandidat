\section{Uppföljning av  planeringsrapport och fortsatt arbete}

Enligt planeringsrapportens tidsplan skulle CAD modellering utföras i arbetsvecka 9, beräkning och simulering i vecka 10 och prototypbygge i vecka 11. Alla dessa tre var redan påbörjade i vecka 4 dock. Detta beror på att bilden av hur projektet ska utföras har förändrats sedan planeringsrapporten.  När projektet befann sig i ett tidigare stadie hölls det abstrakt för att vara öppet för så många olika lösningar som möjligt. Arbetet var anpassat mer för en produktutvecklingsprocess.

 När arbetet inleddes ändrades dock karaktären till att vara mer av ett målbestämt tekniskt arbete med tydligare riktlinjer för vad som ska göras. Mer tyngd lades alltså på att snabbt skapa ett mindre antal mycket klara mål för att kunna påbörja det tekniska arbetet snabbare.

Arbetet med de mekatroniska delmålen kunde påbörjas snabbare än mjukvarudelmålen eftersom resurserna för mekaniken var tillgängliga tidigare. Framtagningen av den mekaniska  designen gick snabbare än planerat på grund av valet att basera prototypen på Meccano. Då fanns möjligheten att experimentera med existerande Meccanodelar utan att behöva motivera inköp med beräkningar. Dessutom blev designen begränsad till Meccanos mått.

Enligt tidsplanen skulle de första 7 veckorna läggas på produktutvecklingsprocessen. Det framgick dock snabbt att ett koncept som tagits fram visade sig vara överlägset alla andra. Det gjorde att ett konceptval snabbt kunde göras och att en vidareutveckling kunde börja tidigare än planerat.

Projektplanen har nu fått mycket klarare upplägg än tidigare med klarare delmål. Genom att betrakta robothanden och styrning av den som ett system och sedan bryta ner detta system i mindre delar på ett tydligare sätt än tidigare blev planeringen också lättare att göra.


Det som kvarstår är att färdigställa den mekaniska delen av handen fullständigt. En av uppgifterna är att slutföra kraftöverföringen mellan styrservona och fingrarna vilket betyder att servona måste monteras på sina respektive platser.
 Hela handen behöver även en slutgiltig färdigställning för att bli stabilare och mer estetiskt tilltalande.

Skelettet till fingrarna är färdiga, det som saknas är greppytor med integrerande trycksensorer som ska konstrueras, tillverkas och monteras på skelettet.

Störst fokus hädanefter kommer att hamna på konstruktionen av den elektroniska delen. Området innefattar konstruktionen av kontrollhandsken och alla de kretsar som tillkommer för att styra handen. När det fullständiga systemet är upprättat kommer regleringen och programeringen vara nästa stora steg.

\subsection{Frågeställningar} 
\begin{itemize}
\item Kontakt med föremål
\end{itemize}

Hur ska rörelsen skötas vid kontakt med föremål? När fingrarna rör sig fritt är det lämpligt att fingrarna rör sig följsamt för att imitera den mänskliga handen som ger styrsignalen. Vid kontakt med föremål är inte fingrarnas direkta rörelsen lika viktig utan istället kraften som verkar på objektet. En mycket liten förändring i styrhanden ger en stor kontaktkraft på objektet. Problemet blir då hur kontrollenheten ska skifta mellan två olika lägen där den i ena läget följer användarens rörelser och den i andra läget skalar ner användarens rörelse för ökad finkänslighet. 

\begin{itemize}
\item Utformning av fingertopparna
\end{itemize}
För att trycksensorerna ska få plats på fingertopparna kan enskilda delar tillverkas separat och monteras på det nuvarande skelettet. Förutom hur de ska designas för att innehålla trycksensorerna är möjliga problem som kan uppkomma hur topparna ska designas för att inte störa den nuvarande handens rörelser, samt samtidigt klara att utföra de mål som är uppsatta.









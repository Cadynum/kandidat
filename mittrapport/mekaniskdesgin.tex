\section{Mekanisk design}
Robothanden består av fyra huvuddelar, två fingrar, en tumme och en handflata.
Med utgångspunkten mänsklig motorik designades handens två identiska fingrar för att få ett människolikt rörelsemönster. Fingrarna har tre leder varav Led 1 och Led 2 är separat kontrollerbara. Led 3 är via ett stag tvångsstyrd av Led 2 för att imitera hur ett mänskligt finger beter sig när handen sluts. Jämfört med det mänskliga fingret saknas en frihetsgrad i Led 1 för vridning av fingret i sidled. En fördel med två separat kontrollerbara leder är att fingrarnas rörelseomfång och funktionella förmåga utökas. 
\begin{figure}[H]
\includegraphics[width=0.5\textwidth]{fingerbild}
\caption{Mekanisk ritning över ett figner.}
\end{figure}
Handens utformning är endast en funktion av hur fingrarna önskas vara positionerade relativt varandra och detta utprovades i CAD-miljö utefter förmågan att utföra de önskade greppen. 
\begin{figure}[H]
\includegraphics[width=0.5\textwidth]{???}
\caption{Hej en viktig bild}
\end{figure}
(skaffa bild)

Led 1 i fingrarna och tummen aktueras via stag medans Led 2 är kopplad till aktuator via en sena.
\begin{figure}[H]
\includegraphics[width=0.5\textwidth]{???}
\caption{Hej en viktig bild}
\end{figure}
(bild senans väg i fingret)
 
För att återföra fingret sitter en spiralfjäder mellan staget och konsol vid Led 1.
Aktuatorer för samtliga leder är Blue bird servon med ett vridmoment på 1.42Nm, en reaktionstid på 0.17 sekunder och vikt på 52 gram
Dessa servon har ett rörelseomfång 60^o åt varje håll och är självreglerande, vilket eliminerar behovet av mätning och återkoppling.
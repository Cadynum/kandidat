%!TEX root = main.tex
\section{Elektronik}

\subsection{Kontrollenhet}


%EMIL: Vilka trycksensorer och flexsensorer?
\subsubsection{Kontrollhandske}
För att kontrollera robothanden används en handske med
snabbhet
säkerhet
prestanda

\begin{figure}[ht]
\includegraphics[width=0.3\textwidth]{img/arduino_micro}
\caption{Ardunio Micro mikrokontroller}
\end{figure}



\subsection{Trådlös överföring}
För att möjliggöra kommunikation trådlöst mellan kontrollhandsken och robothanden används två prototyp-versioner av \emph{Bluetooth Mate Silver}. Enheten är godkänd för Bluetooth klass 2, vilket bl a innebär låg strömförbrukning (genomsnittligt 2.5mW vid aktiv användning) och överföringar på upp till 10m, vilket anses vara tillräckligt för att klara kravet på 5m i praktiska förhållanden.

Bluetooth-protokollet är ett paketförmedlande nätverk, vilket innebär att information skickas i diskreta paket. Varje paket innehåller metadata
\footnote{Data som beskriver information, som mottagaradress och felkorrigerande kod}
och \emph{payload}
\footnote{Datan i sig som ska överföras}.
Bluetooth-enheten har en överföringskapacitet på 115200bps, inkluderat metadata.


\begin{figure}[ht]
\includegraphics[width=0.3\textwidth]{img/bluetooth_mate_silver.jpg}
\caption{Bluetooth Mate Silver enhet}
\end{figure}

\paragraph{Säkerhet}
Bluetooth definierar ett säkerhetsprotokoll där enheter utbyter symmetriska krypteringsnycklar \footnote{AES blabla} antingen med hjälp av en dynamisk ``parningsprocess'', där användaren kontrollerar pinkoder vid första användningen, eller att nycklarna manuellt. Eftersom varken kontrollhandsken eller robothanden har en display för att godkänna pinkoder används den manuella metoden.

Bluetooth är en industristandard vilket leder till att säkerheten är väl testad och beprövad till den mån att den kan anses vara fullgod för projektet, och om industriella tillämpningar i framtiden skulle var önskvärt \cite{btsec}.

\paragraph{Integritet}
 I metadatan finns bl a


\paragraph{extensibility}



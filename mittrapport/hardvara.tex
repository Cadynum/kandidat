\section{Elektronik}

\subsection{Kontrollenhet}
\subsubsection{Sensorhandske}
snabbhet
säkerhet
prestanda

\subsection{Trådlös överföring}
För att möjliggöra kommunikation trådlöst mellan kontrollhansken och robothanden används två prototypversioner av \emph{Bluetooth Mate Silver}. Enheten är klassificierad Bluetooth klass 2, vilket bla innebär låg strömförbrukning (genomsnittligt 2.5mW vid aktiv användning) och överföringar på upp till 10m, vilket anses vara tillräckligt för att klara kravet på 5m i praktiska förhållanden.

Bluetoothprotokollet är ett paketförmedlande nätverk, vilket innebär att information skickas i diskreta paket. Varje paket innehåller metadata\footnote{Data som beskriver paketet i sig, som mottagaradress och felkorrigerande kod} och en \emph{payload}\footnote{Teknisk beskrivning för informationen i sig. Bluetoothenheten har en överföringskapacitet på 115200bps, inkluderat metadata. 

\paragraph{Säkerhet}
Bluetooth definierar ett säkerhetsprotokoll där enheter utbyter symmertriska krypteringsnycklar \footnote{AES blabla} under en  ``parningsprocess''. Eftersom bluetooth är en industristandard för trådlös överföring är säkerheten väl testad och beprövad till den mån att den kan anses vara fullgod för projektet, även om industritella tillämpningar i produktion skulle var önskvärt \cite{MISSING}.

\paragraph{Integritet}
 I metadatan finns bl a 


\paragraph{extensibility}



\begin{figure}[h]
\includegraphics[width=0.3\textwidth]{img/bluetooth_mate_silver.jpg}
\caption{Bluetooth Mate Silver enhet}
\end{figure}
säkerhet
integretet
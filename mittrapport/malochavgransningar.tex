\section{Mål}
För att konkretisera designuppgiften och koppla syftet till mätbara mål har följande konkreta uppgifter för handen tagits fram.
\begin{itemize}
\item Handen ska kunna gripa och lyfta ett mjölkpaket med en vikt av 1kg enligt bilden nedan. 
\item Handen ska kunna gripa och lyfta en mutter av storlek M10 mellan tumme och pekfinger.
\item Handen ska klara av att trycka på en knapp som är placerad på en yta som är vinkelrät mot handflatan.
\item Handen ska kunna lyfta upp en penna.
\item Handen ska kunna lyfta upp en snusdosa.
\item Fingerspetsen ska kunna inta två olika lägen utan att flytta handen. (trycka på två olika knappar)
\item Det ska vara möjligt att mäta trycket som handen påverkar objektet med.
\item Handen skall kunna förhindra att skada görs på kända objekt p.g.a för stora krafter.
(Hänvisa alla mål till greppbilden från Cutkosky )
\item En ovan användare skall efter kalibrering kunna utföra ovanstående mål
\end{itemize}
Målet är att utveckla en hand med tre fingrar (två fingrar, en tumme) till låg kostnad. Målen ska verifieras genom att testa de olika momenten och dokumenteras med videoinspelning.



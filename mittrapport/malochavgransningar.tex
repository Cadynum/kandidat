\section{Mål}
För att konkretisera designuppgiften och koppla syftet till mätbara mål har följande konkreta uppgifter för handen tagits fram.
\begin{itemize}
\item Handen ska kunna gripa och lyfta ett mjölkpaket med en vikt av 1kg. (Grasp 1 \cite{Cutkosky})
\item Handen ska kunna gripa och lyfta en mutter av storlek M10 mellan tumme och pekfinger. (Grasp 9 \cite{Cutkosky})
\item Handen ska klara att att lyfta en last motsvarande ett kg på mitten av ett finger.
\item Handen ska kunna lyfta upp en penna.(Grasp 8 \cite{Cutkosky})
\item Handen ska kunna lyfta upp en snusdosa.
\item Fingerspetsen ska kunna inta två olika lägen utan att flytta handen. (trycka på två olika knappar)
\item Information om trycket som handen påverkar objektet med ska mätas.
\item Handen skall kunna kontrollera trycket som den applicerar på objekt.
(Hänvisa alla mål till greppbilden från Cutkosky )
\item En ovan användare skall efter kalibrering kunna utföra ovanstående mål.
\item Maximal tid för handen att röra sig från maximalt öppen till en knuten näve är en sekund.
\end{itemize}

 Målen ska verifieras genom att testa de olika momenten och dokumenteras med videoinspelning.



\section{Bakgrund}
Robothänder är intressanta ur ett flertal perspektiv. En väl fungerande robothand kan användas som ett mycket effektivt hjälpmedel, inte bara för personer i behov av en protes, utan även i industriella tillämpningar. Monotona, fysiskt tunga, och direkt farliga arbeten kan utföras av en robothand istället för en människa. Det finns också stora vinster i att ha en robothand som kan utföra arbeten i en industriell process som annars måste stoppas för att möjliggöra säker tillgång för en människa eller för att kunna utföra enkla arbeten snabbare och under längre tidsperioder än vad som är möjligt för en människa. Dagens teknik, gör det även möjligt för experter att fjärrstyra robothänder, eller system innehållande dessa, för att utföra komplexa arbeten runtom i världen utan att själva behöva resa runt \cite{Bichi}.

Att med robotteknik efterlikna en så komplex konstruktion som den mänskliga handen är en tekniskt svår utmaning som länge eftersträvats. I tillämpningar är det dock vanligast att en en mindre komplex mekanisk konstruktion används, men den kan ha andra fördelar med avseende på t.ex. styrka eller precision istället. För att undvika att robothanden skadar objekt eller sig själv kan man utnyttja feedback från sensorer, och med lämplig åtgärd hindra att för stora krafter alstras, en risk som är särskilt påtaglig vid fjärrarbete där användaren har begränsad överblick över systemet.

Att designa och bygga en funktionell robothand är en utmaning som spänner över ett flertal ingenjörsmässiga områden. Mekanisk utformning och dimensionering för att kunna uppnå tillräckligt rörelseomfång och styrka, utformning och reglering av det mekatroniska systemet bestående av aktuatorer och sensorer. Därtill behövs datateknik för att hantera säker trådlös överföring av styrsignaler och programmering av kontrollenheter.

Sammantaget gör de möjliga tillämpningsområdena och de många tekniska utmaningarna att Design av robothand är ett väl motiverat kandidatarbete.

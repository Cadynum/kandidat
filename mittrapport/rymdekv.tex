\section{Rymdekvationer}

Som underlag till övriga beräkningar behövs en mattematisk modell över handens möjliga rörelsemönster. Bland annat behövs dessa beräkningar i ett senare stadie för att koppla önskad handposition till en motsvarande servovinkel.  Handens position bestäms av två inmatningsvariabler i form av vinklarna $\alpha$ och $\beta$. $\alpha$ utgår från y-axeln i ett koordinatsystem med origo i punkt P1 (se figur A.4.), medens $\beta$ utgår ifrån y-axeln i ett koordinatsystem med origo i punkt P3. Med andra ord kommer $\beta$ vinken att vara beroende av $\alpha$ då dess koordinat-system är kopplat till denna invariabel. Initialt behövs de vinklar som bestämmer samtliga punkters position beräknas vilket  görs med hjälp av ekvation \eqref{eq:rymdekv1} -- \eqref{eq:rymdekv10}. Ekvation \eqref{eq:rymdekv3} --\eqref{eq:rymdekv6} har tagits fram med hjäp av applicering av sinussattsen, medens \eqref{eq:rymdekv8} baseras på vinkelsummeringsregeln för en triangel. 

\begin{minipage}[t]{0.5\textwidth}
\begin{figure}[H]
\label{img:rymdimg1}
\includegraphics[width=1\textwidth]{img/rymdekv_length}
\caption{Definition av länger.}
\end{figure}
\end{minipage}
\begin{minipage}[t]{0.5\textwidth}
\begin{figure}[H]
\label{img:rymdimg2}
\includegraphics[width=0.93\textwidth]{img/rymdekv_ang}
\caption{Definition av vinklar}
\end{figure}
\end{minipage}


\medskip{}

\begin{equation}
\label{eq:rymdekv1}
L_{1}= L_{11}+L_{12}
\end{equation}

\begin{equation}
\label{eq:rymdekv2}
L_{2}= L_{21}+L_{22}
\end{equation}


\begin{align}
\label{eq:rymdekv3}
L_{11}=\frac{a\sin(\gamma)}{\sin(\theta)}  \\
\label{eq:rymdekv4}
L_{12}=\frac{b\sin(\lambda)}{\sin(\theta)} 
\end{align}


\begin{align}
\label{eq:rymdekv5}
L_{21}=\frac{a\sin(\xi)}{\sin(\theta)} \\
\label{eq:rymdekv6}
L_{12}=\frac{b\sin(\delta)}{\sin(\theta)} 
\end{align}

\begin{figure}[H]
\label{felbild}
\includegraphics[width=0.4\textwidth]{img/kandidat_figur_b}
\caption{Definition av $\gamma$}
\end{figure}

\begin{equation}
\label{eq:rymdekv7}
\gamma = \alpha + \pi + \beta 
\end{equation}

\begin{equation}
\label{eq:rymdekv8}
\xi+ \gamma+ \theta=\pi
\end{equation} 

\begin{equation}
\label{eq:rymdekv9}
\lambda+\delta+\theta=\pi 
\end{equation} 

Ur den mattematiska modellen erhålls nio okända variabler, fem okända vinklar och 4 okända längder vilka löses med ekvation \eqref{eq:rymdekv1} -- \eqref{eq:rymdekv9}. Sammanställningen av dessa ekvationer resulterar i ekvation \eqref{eq:rymdekv10} -- \eqref{eq:rymdekv11}. Då vinklarna $\theta$ och $\gamma$ är beroende av varandra går dessa inte att lösa ut. För att beräkna $\theta$ och $\gamma$ applicerades en iterativ lösningsmodell i matlab. Den antagna itterationsmodellen är stabil med snabb konvergens då problemets komplexitet är relativt låg.  



\begin{align}
\label{eq:rymdekv10}
\theta{} &= \arcsin\left(\frac{a\sin(\pi-\theta-\gamma)+b\sin(\pi-\theta-\lambda )}{L_{2}}\right) \\
\bigskip{}
\label{eq:rymdekv11}
\lambda{} &=\arcsin\left(\frac{L_{1}\sin(\theta)-a\sin(\gamma)}{b}\right) 
\end{align}

\[
\begin{cases}
L_{1}=59 [mm] \\
L_{2}=49 [mm] \\
L_{3}=38 [mm] \\
L_{stag}=49 [mm]\\
a=12 [mm] \\
b=12 [mm] \\
\end{cases}
\]

 

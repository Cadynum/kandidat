\section{Framtida arbete}
Det som kvarstår är att färdigställa handens mekatroniska system. En av uppgifterna är att slutföra kraftöverföringen mellan styrservona och fingrarna vilket betyder att servona måste monteras på sina respektive platser. Då olika servoalternativ varit lämpliga för olika designalternativ av handen har ett iterativt arbete pågått, där slutgiltigt val nyligen gjorts.
Hela handen behöver även en slutgiltig färdigställning när alla komponenter är på plats.

Skelettet till fingrarna är färdiga, det som saknas är greppytor med integrerande trycksensorer som ska konstrueras, tillverkas och monteras på skelettet.

Stort fokus hädanefter kommer att hamna på konstruktionen av den elektroniska delen. Området innefattar konstruktionen av kontrollhandsken och alla de kretsar som tillkommer för att styra handen. När det fullständiga systemet är upprättat kommer regleringen och programeringen vara nästa stora steg.

\subsection{Frågeställningar} 
\begin{itemize}
\item Kontakt med föremål
\end{itemize}

Hur ska rörelsen skötas vid kontakt med föremål? När fingrarna rör sig fritt är det lämpligt att fingrarna rör sig följsamt för att imitera den mänskliga handen som ger styrsignalen. Vid kontakt med föremål är inte fingrarnas direkta rörelse lika viktig utan istället kraften som verkar på objektet. En mycket liten förändring i styrhanden ger en stor kontaktkraft på objektet. Problemet blir då hur kontrollenheten ska skifta mellan två olika lägen där den i ena läget följer användarens rörelser och den i andra läget skalar ner användarens rörelse för ökad finkänslighet. 

\begin{itemize}
\item Utformning av fingertopparna
\end{itemize}
För att trycksensorerna ska få plats på fingertopparna kan enskilda delar tillverkas separat och monteras på det nuvarande skelettet. Förutom hur de ska designas för att innehålla trycksensorerna är möjliga problem som kan uppkomma hur topparna ska designas för att inte störa den nuvarande handens rörelser, samt  klara att utföra de mål som är uppsatta.


\begin{itemize}
\item Input från olika användare
\end{itemize}
Då töjningsresistorerna på styrhandsken ändrar resistans beroende på hur handsken deformeras blir konsekvensen att användare med olika stora händer ger styrsignaler som påverkar handen olika. För att lösa problemet måste det därför finnas ett sätt att kalibrera handsken mellan olika användare.

\begin{itemize}
\item Översätta önskad fingervinkel till servovinkel
\end{itemize}
Då användaren önskar att ställa fingret i en viss vinkel måste detta realiseras genom att servona vrider sig till en vinkel som motsvarar detta. Då servona är nyligen utvalda är kraftöverföringen mellan servo och finger inte fullständigt konstruerad. Det som redan är känt är att sambandet mellan servons vinkel och fingrets vinkel är ickelinjärt. Då kraftöverföringen är fullständigt konstruerad kommer de hittills okända måtten i konstruktionen att kunna sättas in i beräkningar och ett approximativt samband kommer tas fram genom att kurvanpassning med ex.vis minsta kvadrat metoden eller liknande, för att kunna implementera en enklare funktion i Robothandens Arduino enhet.
\section{Beräkningar}
För att kunna dimensionera servona behövdes de krafter som påverkar handen räknas ut. Beräkningarna gjordes genom att frilägga ett enskilt finger och rita ut de moment och krafter som verkar på fingret. Gripdonet belastades enligt två olika greppfall: Vertikal last och omfattning av objekt. 
  För att underlätta beräkningarna gjordes följande antaganden:
   Vid vertikal last belastas fingret belastas med en punktkraft motsvarande ett kg på mitten av den näst yttersta fingerleden. Den yttre fingerleden tas därmed inte med i beräkningarna eftersom den i första hand ska utföra finmotoriska uppgifter där kraven på greppstyrka inte är lika viktiga. 
Fingret ställs i en position där den näst yttersta fingerleden hamnar i ett horisontellt läge och vinkeln mellan en vertikal linje och leden närmast handen är 30 grader.
Vid omfattning av objekt ska ett objekt storleksordning av ett mjölkpaket gripas med tillräcklig kraft för att lyfta upp det.
Eftersom den givna lasten är relativt hög försummas motorernas och strukturens egentyngd. 
    
För att beskriva fingrets läge i rymden och få en uppfattning om området som fingerspetsen kommer att röra sig i beskrevs fingret med matematiska formler. Då fingerdel nummer ett och tre är sammanlänkade med ett stag kommer fingerspetsens läge endast vara beroende av två vinklar och de olika delarnas mått.
För att mata in användaranvisningar användes en resistans som ändrar motstånd beroende på hur den deformeras. Resistansen kopplas i serie med 